\documentclass[12pt,spanish,openany,letterpaper,pagesize]{scrbook}

\usepackage[utf8]{inputenc}
\usepackage[spanish]{babel}%escribir con acentos sin necesidad de comandos \'{} .
\usepackage[sort&compress]{natbib}
\usepackage{fancyhdr}
\usepackage{graphics}
\usepackage{epsfig}
\usepackage{epic}
\usepackage{eepic}
\usepackage{amsmath}
\usepackage{amssymb}
\usepackage{threeparttable}
\usepackage{amscd}
\usepackage{here}
\usepackage{graphicx}
\usepackage{lscape}
\usepackage{tabularx}
\usepackage{subfigure}
\usepackage{longtable}
\usepackage{float}
\usepackage{multirow}

%\usepackage{titlesec}

\usepackage{soul, color}
\usepackage{tikz}
\usepackage{pgfplots}
\usepackage{latexsym}

\usepackage{listings}
\usepackage{xcolor}
\lstset { %
	language=C++,
	basicstyle=\ttfamily,
	keywordstyle=\color{blue},
	commentstyle=\color{olive},
	morekeywords={Cell, Face, Cells_t, Fluid, Equation_Base, Equilibrium_Relation, Interfluid_Interaction, Well, Mesh, Rock, Producer_Well, Injector_Well}
	%backgroundcolor=\color{black!5}, % set backgroundcolor
	%basicstyle=\footnotesize,% basic font setting
}

\usepackage{rotating} %Para rotar texto, objetos y tablas seite. No se ve en DVI solo en PS. Seite 328 Hundebuch
                        %se usa junto con \rotate, \sidewidestable ....


\renewcommand{\theequation}{\thechapter-\arabic{equation}}
\renewcommand{\thefigure}{\textbf{\thechapter-\arabic{figure}}}
\renewcommand{\thetable}{\textbf{\thechapter-\arabic{table}}}


\pagestyle{fancyplain}%\addtolength{\headwidth}{\marginparwidth}
\textheight22.5cm \topmargin0cm \textwidth16.5cm
\oddsidemargin0.5cm \evensidemargin-0.5cm%
\renewcommand{\chaptermark}[1]{\markboth{\thechapter\; #1}{}}
\renewcommand{\sectionmark}[1]{\markright{\thesection\; #1}}
\lhead[\fancyplain{}{\thepage}]{\fancyplain{}{\rightmark}}
\rhead[\fancyplain{}{\leftmark}]{\fancyplain{}{\thepage}}
\fancyfoot{}
\thispagestyle{fancy}%


\addtolength{\headwidth}{0cm}
\unitlength1mm %Define la unidad LE para Figuras
%\mathindent0cm %Define la distancia de las formulas al texto,  fleqn las descentra
\marginparwidth0cm
\parindent0cm %Define la distancia de la primera linea de un parrafo a la margen

%Para tablas,  redefine el backschlash en tablas donde se define la posici\'{o}n del texto en las
%casillas (con \centering \raggedright o \raggedleft)
\newcommand{\PreserveBackslash}[1]{\let\temp=\\#1\let\\=\temp}
\let\PBS=\PreserveBackslash

%Espacio entre lineas
\renewcommand{\baselinestretch}{1.1}

%Neuer Befehl f\"{u}r die Tabelle Eigenschaften der Aktivkohlen
\newcommand{\arr}[1]{\raisebox{1.5ex}[0cm][0cm]{#1}}

%Neue Kommandos
\usepackage{Befehle}
%Inicio del documento. Tener en cuenta que hay archivos auxiliares

%%%%%%%%%%%%%%dont hyphenize%%%%%%%%%%%%%%%%%%%%%%%%%%%%%%%%
\tolerance=1
\emergencystretch=\maxdimen
\hyphenpenalty=10000
\hbadness=10000

\begin{document}
\pagenumbering{roman}
%\newpage
%\setcounter{page}{1}
\begin{center}
\begin{figure}
\centering%
\epsfig{file=HojaTitulo/EscudoUN,scale=1}%
\end{figure}
\thispagestyle{empty} \vspace*{0.0cm} \textbf{\LARGE
Un modelo ejecutable para la simulación multi-f\'{\i}sica de procesos de recobro mejorado en yacimientos de petr\'{o}leo basado en esquemas preconceptuales}\\[5.3cm]
\Large\textbf{Steven Velásquez Chancí}\\[5.3cm]
\small Universidad Nacional de Colombia\\
Facultad de Minas, Departamento de Ciencias de la Computación y la Decisión\\
Medellín, Colombia\\
2019\\
\end{center}

\newpage
\begin{center}
\thispagestyle{empty} \vspace*{0cm} \textbf{\LARGE
Un modelo ejecutable para la simulación multi-f\'{\i}sica de procesos de recobro mejorado en yacimientos de petr\'{o}leo basado en esquemas preconceptuales}\\[2.0cm]
\Large\textbf{Steven Velásquez Chancí}\\[2.0cm]
\small Tesis de investigación presentada como requisito parcial para optar al
t\'{\i}tulo de:\\
\textbf{Magíster en Ingeniería de Sistemas}\\[2.0cm]
Director:\\
Ph.D. Juan Manuel Mejía Cárdenas\\
Co-Director:\\
Ph.D. Carlos Mario Zapata Jaramillo\\[2.0cm]
L\'{\i}nea de Investigaci\'{o}n:\\
Ingeniería de Software\\
Grupos de Investigaci\'{o}n:\\
Dinámicas de Flujo y Transporte en medios porosos\\
Lenguajes Computacionales\\[2.0cm]
Universidad Nacional de Colombia\\
Facultad de Minas, Departamento de Ciencias de la Computación y la Decisión\\
Medellín, Colombia\\
2019\\
\end{center}

\newpage{\pagestyle{empty}\cleardoublepage}

\newpage
\thispagestyle{empty} \textbf{}\normalsize
\\\\\\%
~\\[4.0cm]

\begin{flushright}
\begin{minipage}{8cm}
    \noindent
        \small
        ~\\[2.0cm]
        A mi familia, amigos y a todos los que\\
        estuvieron pendientes de este proceso.\\
        ¡Se logró!
        
        %La preocupaci\'{o}n por el hombre y su destino siempre debe ser el
        %inter\'{e}s primordial de todo esfuerzo t\'{e}cnico. Nunca olvides esto
        %entre tus diagramas y ecuaciones.\\\\
        %Albert Einstein\\
\end{minipage}
\end{flushright}

\chapter*{Agradecimientos}
\addcontentsline{toc}{chapter}{\numberline{}Agradecimientos}
Agradezco a mi Director Juan Manuel Mejía Cardenas por su supervisión y confianza durante este proceso, a mi Codirector Carlos Mario Zapata Jaramillo por su guía y por saberme centrar cuando lo he requerido, a Juan David Valencia Londoño, cuya constante disposición para solucionar mis dudas me ayudaron a llevar un avance continuo, a Felipe Ospina Montoya, cuyo apoyo ha sido invaluable, y a Paola Andrea Nore\~{n}a Cardona, por su fé, apoyo, asesoría y acompañamiento constante durante el desarrollo de esta Tesis.\\

Finalmente, agradezco al Departamiento de Ciencias de la Computación y la Decisión por financiar mis estudios por medio de la beca de Facultad y a la alianza formada por el Departamento Administrativo de Ciencia, Tecnología e Innovación (COLCIENCIAS), la Agencia Nacional de Hidrocarburos (ANH) y la Universidad Nacional de Colombia por el financiamiento del ``Plan Nacional para el Potenciamiento de la Tecnología CEOR con Gas Mejorado Químicamente'' bajo el acuerdo 273-2017, dentro del cual se enmarca la investigación asociada a esta Tesis de Maestría.

\newpage{\pagestyle{empty}\cleardoublepage}

\chapter*{Resumen}
\addcontentsline{toc}{chapter}{\numberline{}Resumen}

La simulación de procesos de recobro mejorado se rige por las leyes en las que se describe el transporte de fluidos en medios porosos. Existen múltiples propuestas de elaboración de \textit{frameworks} y simuladores para procesos de recobro mejorado. Sin embargo, carecen de trazabilidad de conceptos, procesos y de representación de eventos que surgen de la física. En los Esquemas Preconceptuales (EP) se incluye toda la estructura de un dominio de aplicación y, también se pueden representar los procesos que se dan en tal dominio. Estos, aportan cohesión, consistencia y trazabilidad entre conceptos y procesos. Por lo que, en esta Tesis de Maestría se propone un modelo ejecutable para la simulación de procesos de recobro mejorado basado en esquemas preconceptuales. El modelo ejecutable se valida con un caso de estudio. Los resultados del modelo ejecutable se ajustan a los datos reportados en la literatura. El modelo ejecutable propuesto permite trazar consistentemente los conceptos, procesos y eventos presentes en la simulación de procesos de recobro mejorado.\\


\textbf{\small Palabras clave: Modelo Ejecutable, Esquemas Preconceptuales, Simulación de Yacimientos de Petróleo, Procesos de Recobro Mejorado, Representación Basada en Eventos}.\\

%El resumen es una presentaci\'{o}n abreviada y precisa (la NTC 1486 de 2008 recomienda revisar la norma ISO 214 de 1976). Se debe usar una extensi\'{o}n m\'{a}xima de 12 renglones. Se recomienda que este resumen sea anal\'{\i}tico, es decir, que sea completo, con informaci\'{o}n cuantitativa y cualitativa, generalmente incluyendo los siguientes aspectos: objetivos, dise\~{n}o, lugar y circunstancias, pacientes (u objetivo del estudio), intervenci\'{o}n, mediciones y principales resultados, y conclusiones. Al final del resumen se deben usar palabras claves tomadas del texto (m\'{\i}nimo 3 y m\'{a}ximo 7 palabras), las cuales permiten la recuperaci\'{o}n de la informaci\'{o}n.\\


%\textbf{\small Palabras clave: (m\'{a}ximo 10 palabras, preferiblemente seleccionadas de las listas internacionales que permitan el indizado cruzado)}.\\

%A continuaci\'{o}n se presentan algunos ejemplos de tesauros que se pueden consultar para asignar las palabras clave, seg\'{u}n el \'{a}rea tem\'{a}tica:\\

%\textbf{Artes}: AAT: Art y Architecture Thesaurus.

%\textbf{Ciencias agropecuarias}: 1) Agrovoc: Multilingual Agricultural Thesaurus - F.A.O. y 2)GEMET: General Multilingual Environmental Thesaurus.

%\textbf{Ciencias sociales y humanas}: 1) Tesauro de la UNESCO y 2) Population Multilingual Thesaurus.

%\textbf{Ciencia y tecnolog\'{\i}a}: 1) Astronomy Thesaurus Index. 2) Life Sciences Thesaurus, 3) Subject Vocabulary, Chemical Abstracts Service y 4) InterWATER: Tesauro de IRC - Centro Internacional de Agua Potable y Saneamiento.

%\textbf{Tecnolog\'{\i}as y ciencias m\'{e}dicas}: 1) MeSH: Medical Subject Headings (National Library of Medicine's USA) y 2) DECS: Descriptores en ciencias de la Salud (Biblioteca Regional de Medicina BIREME-OPS).

%\textbf{Multidisciplinarias}: 1) LEMB - Listas de Encabezamientos de Materia y 2) LCSH- Library of Congress Subject Headings.\\

%Tambi\'{e}n se pueden encontrar listas de temas y palabras claves, consultando las distintas bases de datos disponibles a trav\'{e}s del Portal del Sistema Nacional de Bibliotecas\footnote{ver: www.sinab.unal.edu.co}, en la secci\'{o}n "Recursos bibliogr\'{a}ficos" opci\'{o}n "Bases de datos".\\

\chapter*{Abstract}
\addcontentsline{toc}{chapter}{\numberline{}Abstract}
Enhanced oil recovery (EOR) processes simulation is governed by mass conservation laws. In such laws, flow, accumulation, sources and sinks phenomena in porous media are described. Multiple proposals for framework and simulation elaboration have been defined. However, they lack concepts and processes tracing and event representation for physical phenomena. Preconceptual Schemas (PS) are used for including the complete structure of an application domain and representing processes emerging in such a domain. Cohesion, consistency, and tracing between concepts and processes is obtained by using PS. In this MSc. Thesis an executable model for enhanced oil recovery processes simulation based on preconceptual schemas is proposed. The executable model is validated by running a study case. The results are in accordance with data reported in the literature. The proposed executable model allows for tracing consistently the concepts, processes, and events, which are present in EOR processes simulation. \\

\textbf{\small Palabras clave: Executable Models, Preconceptual Schemas, Oil Reservoir Simulation, Enhanced Oil Recovery Processes, Event based Representation}.\\

%Es el mismo resumen pero traducido al ingl\'{e}s. Se debe usar una extensi\'{o}n m\'{a}xima de 12 renglones. Al final del Abstract se deben traducir las anteriores palabras claves tomadas del texto (m\'{\i}nimo 3 y m\'{a}ximo 7 palabras), llamadas keywords. Es posible incluir el resumen en otro idioma diferente al espa\~{n}ol o al ingl\'{e}s, si se considera como importante dentro del tema tratado en la investigaci\'{o}n, por ejemplo: un trabajo dedicado a problemas ling\"{u}\'{\i}sticos del mandar\'{\i}n seguramente estar\'{\i}a mejor con un resumen en mandar\'{\i}n.\\[2.0cm]
%\textbf{\small Keywords: palabras clave en ingl\'{e}s(m\'{a}ximo 10 palabras, preferiblemente seleccionadas de las listas internacionales que permitan el indizado cruzado)}\\

\renewcommand{\tablename}{\textbf{Tabla}}
\renewcommand{\figurename}{\textbf{Figura}}
\renewcommand{\listtablename}{Lista de Tablas}
\renewcommand{\listfigurename}{Lista de Figuras}
\renewcommand{\contentsname}{Contenido}

%\newcommand{\clearemptydoublepage}{\newpage{\pagestyle{empty}\cleardoublepage}}
\cleardoublepage
\addcontentsline{toc}{chapter}{Contenido} % para que aparezca en el indice de contenidos
\tableofcontents % indice de contenido

\addcontentsline{toc}{chapter}{Lista de figuras} % para que aparezca en el indice de contenidos
\listoffigures % indice de figuras

\addcontentsline{toc}{chapter}{Lista de tablas} % para que aparezca en el indice de contenidos
\listoftables % indice de tablas


%\include{Tab_Simbolos/TabSimbolosMSc}
%\include{Resumen}%\newcommand{\clearemptydoublepage}{\newpage{\pagestyle{empty}\cleardoublepage}}
\pagenumbering{arabic}
% !TeX spellcheck = es
\chapter{Introducci\'{o}n}
%En la introducci\'{o}n, el autor presenta y se\~{n}ala la importancia, el origen (los antecedentes te\'{o}ricos y pr\'{a}cticos), los objetivos, los alcances, las limitaciones, la metodolog\'{\i}a empleada, el significado que el estudio tiene en el avance del campo respectivo y su aplicaci\'{o}n en el \'{a}rea investigada. No debe confundirse con el resumen y se recomienda que la introducci\'{o}n tenga una extensi\'{o}n de m\'{\i}nimo 2 p\'{a}ginas y m\'{a}ximo de 4 p\'{a}ginas.\\
%
%La presente plantilla maneja una familia de fuentes utilizada generalmente en LaTeX, conocida como Computer Modern, espec\'{\i}ficamente LMRomanM para el texto de los p\'{a}rrafos y CMU Sans Serif para los t\'{\i}tulos y subt\'{\i}tulos. Sin embargo, es posible sugerir otras fuentes tales como Garomond, Calibri, Cambria, Arial o Times New Roman, que por claridad y forma, son adecuadas para la edici\'{o}n de textos acad\'{e}micos.\\
%
%La presente plantilla tiene en cuenta aspectos importantes de la Norma T\'{e}cnica Colombiana - NTC 1486, con el fin que sea usada para la presentaci\'{o}n final de las tesis de maestr\'{\i}a y doctorado y especializaciones y especialidades en el \'{a}rea de la salud, desarrolladas en la Universidad Nacional de Colombia.\\
%
%Las m\'{a}rgenes, numeraci\'{o}n, tama\~{n}o de las fuentes y dem\'{a}s aspectos de formato, deben ser conservada de acuerdo con esta plantilla, la cual esta dise\~{n}ada para imprimir por lado y lado en hojas tama\~{n}o carta. Se sugiere que los encabezados cambien seg\'{u}n la secci\'{o}n del documento (para lo cual esta plantilla esta construida por secciones).\\
%
%Si se requiere ampliar la informaci\'{o}n sobre normas adicionales para la escritura se puede consultar la norma NTC 1486 en la Base de datos del ICONTEC (Normas T\'{e}cnicas Colombianas) disponible en el portal del SINAB de la Universidad Nacional de Colombia\footnote{ver: www.sinab.unal.edu.co}, en la secci\'{o}n "Recursos bibliogr\'{a}ficos" opci\'{o}n "Bases de datos".  Este portal tambi\'{e}n brinda la posibilidad de acceder a un instructivo para la utilizaci\'{o}n de Microsoft Word y Acrobat Professional, el cual est\'{a} disponible en la secci\'{o}n "Servicios", opci\'{o}n "Tr\'{a}mites" y enlace "Entrega de tesis".\\
%
%La redacci\'{o}n debe ser impersonal y gen\'{e}rica. La numeraci\'{o}n de las hojas sugiere que las p\'{a}ginas preliminares se realicen en n\'{u}meros romanos en may\'{u}scula y las dem\'{a}s en n\'{u}meros ar\'{a}bigos, en forma consecutiva a partir de la introducci\'{o}n que comenzar\'{a} con el n\'{u}mero 1. La cubierta y la portada no se numeran pero si se cuentan como p\'{a}ginas.\\
%
%Para trabajos muy extensos se recomienda publicar m\'{a}s de un volumen. Se debe tener en cuenta que algunas facultades tienen reglamentada la extensi\'{o}n m\'{a}xima de las tesis  o trabajo de investigaci\'{o}n; en caso que no sea as\'{\i}, se sugiere que el documento no supere 120 p\'{a}ginas.\\
%
%No se debe utilizar numeraci\'{o}n compuesta como 13A, 14B \'{o} 17 bis, entre otros, que indican superposici\'{o}n de texto en el documento. Para resaltar, puede usarse letra cursiva o negrilla. Los t\'{e}rminos de otras lenguas que aparezcan dentro del texto se escriben en cursiva.\\

%%% Importancia de la investigación %%%
Oil reservoir simulation has proven an usefull toll for predicting reserves and production along the years. The simulation of such a problem consists of solving a coupled set of mass balance equations across a domain (reservoir, geometry - geological model). Therefore, the creation of oil reservoir simulators is in the scietific software research area (Rewrite). 

Oil reservoir simulation consists of solving a set of coupled mass or moles balance equations, these equations are non-linear and need adecuate treatment in order to have a linear system that converges.

Since the natural production is no longer maintainable, techniques of enhanced oil recovery (EOR) have been developed in order to mantain or even improve the recovery factor. These techniques involve the injection of chemicals that affect the rock and fluids properties making feasible to change the oil mobility and residual saturations... The EOR processes add new equations to the system that make the problem even bigger. Many authors have addressed this problem by making general flow simulation frameworks. Those frameworks implement the general workflow of solving the coupled set of equations generated by the phenomena in the reservoir.

Some efforts have been done in the scientific software representation. Nore\~{a} et al. extend the preconceptual schema syntax defined by Zapata, 2007. for taking into account the elements needed in the scientific software context. Chaverra, 2011 includes cycles and conditional selection in the preconcpetual schema. Calle, 2017 defines design patterns in the context of scientific software using preconceptual schemas also extending its syntax.


The existing frameworks vary in implementation, even though they apply the same techniques. This is due to the fact that the design desitions are delegated to the programmer, which is an expert of flow in porous media simulation. Little effort has been done in representing the domain of reservoir simulation as is, including both dynamics and structure in the same representation. The existing studies in oil reservoir simulation domain representation lack of grouping the structural design with the dynamical behaviour. Others implement directly a solution of the set of differential equations for the specific study case. The problem knowledge is not shareable. The representations existing only account for the structural or, exclusively the dynamical behavior of the tool they developed. The use of the concepts lacks generality. Even though the formal definition of the differential equations, they lack information of constitutive equations. 

In this thesis we propose an event based representation of the enhanced oil recovery simulation using preconceptual schemas. In order to do that, For this purpose. we sizas sizas sizas describe the black oil simulation domain in the preconceptual schema syntax, later we define a generic component with variable kinetical behavior and with the capacity to change the flow properties in each phase.

The developed model couples the models used for an enhanced oil recovery process in a preconceptual schema that represents adecuately the oil reservoir simulation domain, the representation is validated with the SPE Comparative solution project having accordance with the reported results.
%\chapter{Cap\'{\i}tulo 1}
%Los cap\'{\i}tulos son las principales divisiones del documento. En estos, se desarrolla el tema del documento. Cada cap\'{\i}tulo debe corresponder a uno de los temas o aspectos tratados en el documento y por tanto debe llevar un t\'{\i}tulo que indique el contenido del cap\'{\i}tulo.\\
%
%Los t\'{\i}tulos de los cap\'{\i}tulos deben ser concertados entre el alumno y el director de la tesis  o trabajo de investigaci\'{o}n, teniendo en cuenta los lineamientos que cada unidad acad\'{e}mica brinda. As\'{\i} por ejemplo, en algunas facultades se especifica que cada cap\'{\i}tulo debe corresponder a un art\'{\i}culo cient\'{\i}fico, de tal manera que se pueda publicar posteriormente en una revista.\\
%
%\section{Subt\'{\i}tulos nivel 2}
%Toda divisi\'{o}n o cap\'{\i}tulo, a su vez, puede subdividirse en otros niveles y s\'{o}lo se enumera hasta el tercer nivel. Los t\'{\i}tulos de segundo nivel se escriben con min\'{u}scula al margen izquierdo y sin punto final, est\'{a}n separados del texto o contenido por un interlineado posterior de 10 puntos y anterior de 20 puntos (tal y como se presenta en la plantilla).\\
%
%\subsection{Subt\'{\i}tulos nivel 3}
%De la cuarta subdivisi\'{o}n en adelante, cada nueva divisi\'{o}n o \'{\i}tem puede ser se\~{n}alada con vi\~{n}etas, conservando el mismo estilo de \'{e}sta, a lo largo de todo el documento.\\
%
%Las subdivisiones, las vi\~{n}etas y sus textos acompa\~{n}antes deben presentarse sin sangr\'{\i}a y justificados.\\
%
%\begin{itemize}
%\item En caso que sea necesario utilizar vi\~{n}etas, use este formato (vi\~{n}etas cuadradas).
%\end{itemize}

\chapter{Marco teórico}
%
%\section{Black Oil Model}

\section{Simulación de Yacimientos de Petróleo}
La simulación de yacimientos de petróleo es un proceso en el cual se resuelve un conjunto de ecuaciones diferenciales parciales para un dominio dado. La simulación involucra la elección de modelos, métodos de discretización, tanto en el tiempo como en el espacio; métodos de solución de sistemas no lineales, al igual que métodos de solución de sistemas lineales y precondicionamiento de los mismos. Estas elecciones tienen un impacto en las capacidades de solución de ecuaciones que surgen de los diferentes problemas físicos, además del costo computacional implicado.

\subsection{Modelo Black Oil Extendido}

El \textit{Black Oil Model} (BOM) es un modelo de transporte simultáneo de tres fluidos en el que se asume que los hidrocarburos se distribuyen en un gas y un aceite en barril a condiciones de presión y temperatura estándar \citep{jamal2006petroleum, chen2007reservoir, ertekin2001basic}. Este modelo considera que puede haber una transferencia de masa en equilibrio desde aceite al gas, lo que se denomina ``Gas disuelto''. Adicionalmente, en el modelo \textit{Black Oil} extendido, se considera una transferencia de masa desde el gas al aceite, lo que se denomina ``aceite volatilizado''.

\begin{align}
\label{ec:aceite}
\text{aceite: }&\frac{\partial}{\partial t} \left[ \phi \left( \frac{S_{o}}{B_{o}} + \frac{R_{v} S_{g}}{B_{g}} \right) \right]
+ \nabla \cdot \left( \frac{1}{B_{o}} \vec{v_{o}} + \frac{R_{v}}{B_{g}} \vec{v_{g}} \right) + Q_{o} = 0 \\
\label{ec:gas}
\text{gas: }&\frac{\partial}{\partial t} \left[ \phi \left( \frac{S_{g}}{B_{g}} + \frac{R_{s} S_{o}}{B_{o}} \right) \right]
+ \nabla \cdot \left( \frac{1}{B_{g}} \vec{v_{g}} + \frac{R_{s}}{B_{o}} \vec{v_{o}} \right) + Q_{g} = 0\\
\label{ec:agua}
\text{agua: }&\frac{\partial}{\partial t} \left[\phi \left( \frac{S_{w}}{B_{w}} \right) \right] + \nabla \cdot \left( \frac{1}{B_{w}} \vec{v_{w}} \right) + Q_{w} = 0
\end{align}
donde $\vec{v_{p}}$ corresponde la ley de Darcy para el fluido $p = \left\lbrace o:\text{ aceite}, g:\text{ gas}, w:\text{ agua} \right\rbrace $:
\begin{equation}
\vec{v_{p}}=\frac{\mathbb{K} kr_{p}}{\mu_{p} } \nabla{\Phi_{p,f}}
\end{equation}
\subsection{Problema de Valores Iniciales}
%
\subsection{Modelamiento de Pozos}
%
%
\subsection{Discretización}
%Algebraic Equations

\subsection{Método de Newton-Raphson}
%
\section{Procesos de Recobro Mejorado}

\subsection{Modelamiento del Químico}

\section{Esquemas Preconceptuales}

Los Esquemas Preconceptuales (EP) son representaciones intermedias entre el lenguaje natural y un esquema conceptual o un lenguaje formal. Estos esquemas contienen todo el dominio de aplicación de un interesado, y por tanto, sirven para establecer un punto común de entendimiento entre un interesado y un analista de software \citep{zapata2007phd}.

Los EP se desarrollan con la idea de mantener la coherencia y consistencia entre el discurso del interesado y el software desarrollado. 

\subsection{Elementos del Esquema Preconceptual}
\cite{zapata2012unc} define los elementos del EP tal como se observa en la figura \ref{fig:InitialPS} para la representación del dominio del interesado. Estos elementos se dividen en cuatro categorías: nodos, relaciones, enlaces, y agrupadores.\\

\begin{figure}[h]
	\centering%
	\includegraphics[scale=0.51]{Fig/ElementosDelEP.pdf}%
	\caption[Elementos del EP.]{Elementos del EP. Tomado de \citep{zapata2012unc}.} \label{fig:InitialPS}
\end{figure}

\subsubsection{Nodos}
\begin{itemize}
	\item \textbf{Concepto}:
	Los conceptos son sustantivos o sintagmas nominales que representan un actor u objeto dentro del dominio del interesado, se subclasifican en conceptos clase y conceptos hoja, según su jerarquía \citep{zapata2007phd,zapata2012unc}. %En la figura \ref{fig:PS_Concept} se muestra la representación. Un ejemplo de su uso se presenta en la figura \ref{fig:ej_concept}.
	
	\item \textbf{Condicional}: Los condicionales establecen cuando ejecutar una relación dinámica o una especificación a partir de una expresión lógica formada por conceptos o variables, operadores, y valores o parámetros \citep{zapata2007phd,zapata2012unc}. %como se muestra en la figura \ref{fig:}
	
	\item \textbf{Operador}: Los operadores son símbolos lógicos o matemáticos que sirven para formar expresiones a evaluar \citep{zapata2012unc}.
	
	\item \textbf{Asignación}: Sirve para asignar el valor que resulta de una expresión matemática o lógica \citep{zapata2012unc}.
	
	\item \textbf{Atributo Compuesto}: Sirve para representar el acceso a un concepto hoja o atributo desde su concepto clase \citep{zapata2007phd}.
\end{itemize}

\subsubsection{Relaciones}
\begin{itemize}
	\item \textbf{Relación Estructural}: Relaciones permanentes entre dos conceptos, se asocian a los verbos ``ser'' o ``tener'' y establecen generalización o agregación, respectivamente \citep{zapata2007phd,zapata2012unc}.
	\item \textbf{Relación Dinámica}: Se asocian a verbos que denotan acción u operaciones que modifican el dominio de estudio, establecen relaciones transitorias entre el concepto ejecutor de la acción con el concepto objeto de dicha acción \citep{zapata2007phd,zapata2012unc}.
	\item \textbf{Relación Eventual}: Se relaciona con un verbo que denota ocurrencia \citep{zapata2012unc,norena2018Ling}.
\end{itemize}

\subsubsection{Enlaces}
\begin{itemize}
	\item \textbf{Conexión}: Es una flecha unidireccional que sirve para conectar conceptos con relaciones dinámicas o estructurales \citep{zapata2007phd,zapata2012unc}.
	\item \textbf{Implicación}: Es una línea continua y dirigida que sirve para indicar una relación causa-efecto u orden entre relaciones dinámicas, condicionales o eventos. \citep{zapata2007phd,zapata2012unc}. 
	\item \textbf{Línea}: Sirve para conectar un concepto a una nota o instancia \citep{zapata2007phd,zapata2012unc}.
	\item \textbf{Conector de Operador}: Sirve para conectar un valor, concepto, atributo compuesto u otros operadores, a un operador \citep{zapata2012unc}. 
	\item \textbf{Conjunción/Disyunción}: Sirve para agrupar o bifurcar implicaciones, estableciendo una causalidad conjunta o múltiples efectos \citep{zapata2007phd,zapata2012unc}. 
\end{itemize}

\subsubsection{Agrupadores}
\begin{itemize}
	\item \textbf{Nota o Instancia}: Sirve para limitar los valores para un concepto a un conjunto predefinido \citep{zapata2007phd,zapata2012unc}.
	\item \textbf{Especificación}: Sirve para agrupar un conjunto de operaciones que describen una relación dinámica o eventual \citep{zapata2012unc}.
	\item \textbf{Marco}: Sirve para asociar múltiples relaciones dinámicas a una responsabilidad o para agrupar conceptos \citep{zapata2012unc}.
	\item \textbf{Restricción}: Sirve para establecer una condición sobre una especificación de operaciones \citep{zapata2012unc}. Adicionalmente, se usa para establecer ciclos sobre conceptos o condiciones \citep{JChaverra}.
	\item \textbf{Evento}: Es una ocurrencia que habilita cambios de estado en los procesos  \citep{zapata2013Eventos}.
\end{itemize}

\subsection{Esquemas Preconceptuales en el contexto del software científico}

\cite{JCalle, norena2018det} descubren la capacidad del EP para representar aplicaciones en el contexto del software científico. Para ello, definen elementos adicionales que permiten modelar dominios de mayor complejidad. En esta tesis de maestría se usan las condiciones iniciales, conceptos tipo arreglo, parámetros, variables, vectores, operadores predefinidos, operador ``push'', operador ``type'',  y funciones definidas por el analista. En la figura \ref{fig:NewElements} se presenta la representación de los elementos previamente mencionados. A continuación, se explican los elementos adicionales que se usan en el desarrollo de esta tesis.

\begin{figure}[h]
	\centering%
	\includegraphics[width=0.9\linewidth]{Fig/NuevosElementosDelEP.pdf}%
	\caption[Elementos para la representación de Software Científico.]{Elementos para la representación de Software Científico. Los autores a partir de \citep{JCalle,norena2018det}.} \label{fig:NewElements}
\end{figure}

\subsubsection{Condiciones Iniciales}
Las condiciones iniciales son una especificación, de variables y parámetros globales, que se conoce desde el inicio de la simulación \citep{norena2018det}. Un ejemplo de uso se presenta en la figura \ref{fig:EjInitialConditions}.

\begin{figure}[h]
	\centering%
	\includegraphics[width=0.9\linewidth]{Fig/EjInitialConditions.pdf}%
	\caption[Condiciones Iniciales.]{Condiciones Iniciales. Los autores.} \label{fig:EjInitialConditions}
\end{figure}

\subsubsection{Concepto tipo arreglo}
Los conceptos tipo arreglo permiten almacenar de manera permanente múltiples valores, y a su vez, iterar sobre ellos \citep{JCalle}. En esta tesis de maestría se usan conceptos tipo arreglos multidimensionales. En la figura \ref{fig:EjArrConcept} se expone un ejemplo de su uso.

\begin{figure}[h]
	\centering%
	\includegraphics[scale=1]{Fig/EjArrConcepts.pdf}%
	\caption[Conceptos tipo arreglo.]{Conceptos tipo arreglo. Los autores.} \label{fig:EjArrConcept}
\end{figure}

\subsubsection{Parámetro}
Los parámetros se usan para almacenar constantes o definir entradas en la especificación de relaciones dinámicas, especificaciones de tipo marco y, funciones, que reciben múltiples argumentos o parámetros \citep{JCalle, norena2018det}. Se expone un ejemplo de uso en \ref{fig:EjParameter}.\\

\begin{figure}[h]
	\centering%
	\includegraphics[width=0.5\linewidth]{Fig/EjParametro.pdf}%
	\caption[Ejemplo de parámetros.]{Ejemplo de parámetros. Los autores.} \label{fig:EjParameter}
\end{figure}

\subsubsection{Variable Independiente}
Las variables independientes permiten almacenar valores durante la ejecución de una especificación sin estar acopladas a un concepto. Si se definen en las condiciones iniciales, se pueden usan de manera global durante toda la simulación \citep{norena2018det}. En la figura \ref{fig:EjInitialConditions} se pueden ver ejemplos de variables independientes.\\

\subsubsection{Vector independiente}
Los vectores independientes cumplen la misma tarea y propiedades de las variables independientes pero permiten almacenar más de un valor \citep{norena2018det}. Un ejemplo de vector independiente se presenta en \ref{fig:EjVector}.

\begin{figure}[h]
	\centering%
	\includegraphics[width=0.5\linewidth]{Fig/EjVectorIndependiente.pdf}%
	\caption[Ejemplo de vector independiente.]{Ejemplo de vector independiente. Los autores.} \label{fig:EjVector}
\end{figure}

\subsubsection{Operadores Predefinidos}
Son funciones algebraicas y trigonométricas predefinidas que se pueden usar como operadores en el EP \citep{JCalle}.

\subsubsection{Operador Push}
El operador Push sirve para insertar, valores, conceptos o parámetros dentro de un concepto tipo arreglo, el elemento que se inserta queda en la última posición del arreglo  \citep{JCalle}.
\subsubsection{Operador Type}
El operador Type tiene dos versiones, una como operador de asignación y otro como operador de información. En el caso de la asignación, sirve para otorgarle el tipo de una subclase a un concepto. Mientras que en el de información, consulta si el tipo del concepto corresponde con el tipo de una subclase definida \citep{JCalle}.
\subsubsection{Funciones definidas por el analista}
Las funciones definidas por el analista son especificaciones reutilizables en el EP a modo de un operador personalizado cuyo nombre define el analista. Éstas funciones llevan en su especificación un concepto ``return'' que corresponde al valor que devuelve la función al ser usada como operador \citep{JCalle}. En la figura \ref{fig:Potencial} se presenta un ejemplo de definición y uso de una función definida por el analista.

\begin{figure}[h]
	\centering%
	\includegraphics[scale=0.8]{Fig/EjFuncion.pdf}%
	\caption[Ejemplo de función, cálculo del potencial.]{Ejemplo de función, cálculo del potencial. Los autores.} \label{fig:Potencial}
\end{figure}


\include{Kap3/Kap3}
%\chapter{Cap\'{\i}tulo 3}
\chapter{Solution Proposal}

In this section we propose one element as an extension for Preconceptual Schemas(PS) which aid the understanding of diverse elements in the oil reservoir simulation domain. In addition, we present further description of the concepts stated in the theoretical framework, with their respective representation in the elaborated PS.
\\
This section is structured as follows: in section \ref{sec:PSNew} we present the added elements to PS and their usage in our represented domain. In section \ref{sec:PS_EOR} we propose the representation of structural and dynamical behavior of each developed concept in the theoretical framework using PS.

\section{Added elements to Preconceptual Schemas}\label{sec:PSNew}
\subsection{Analyst defined subroutines}\label{sec:PS_ADS}
Analyst defined subroutines are analyst defined functions as proposed by (ref Calle) without the return argument. They use global elements and parameters of the subroutine definition. They are defined for re-using dynamical behavior elements which appear more than once in the PS. Names of both subroutines and functions must differ from operators predefined in the PS. Graphic symbol used for subroutines is the same as used for operators and functions. In figure \# we present graphical representation of analyst defined subroutines.

\section{Conceptualization}
%
\subsection{Mesh}
%
\subsection{Rock}
%
\subsection{Phase}
%
\subsection{Inter-phase interaction}
%
\subsection{Component}
%
\subsection{Equlibrium Relation}
%
\subsection{Well}

\section{PS Representation of Enhanced Oil Recovery Simulation}\label{sec:PS_EOR}
In this section we propose a PS representation for enhanced oil recovery simulation, we couple a black oil model discretized using finite volumes method with the theoretical framework developed the previous chapter. We mapped each term in the resultant equations to their respective concepts and how they are linked together. The complete representation is shown in \ref{fig:PSComplete}.\\
\begin{figure}[h]
\centering%
\includegraphics[width=0.9\linewidth]{Kap4/MultifasicoSinPozos.pdf}%
\caption{Complete PS Representation for EOR Processes} \label{fig:PSComplete}
\end{figure}

The rest of this section is as follows: In section \ref{sec:PS_Mesh} we present the Mesh concept as a collection of cells with additional elements needed for calculating the attributes of each cell. Furthermore, we develop the dynamical relationship ``Geomodeler defines Mesh'' as an interaction of the role ``Geomodeler'' with atomic\footnote{atomic as is stated by \cite{AG01} (Aca debe ir Zapata)} dynamical relationships. In section \ref{sec:PS_Rock} we present the Rock concept with its attributes and initial characterization. In section \ref{sec:PS_Phase} ... In section \ref{sec:PS_Interphase} ... In section \ref{sec:PS_Equilibrium} we define the partition coefficients as relations between two phases, one contributing mass and another receiving mass in the mass balance equation. In section \ref{sec:PS_Well}  

\subsection{Mesh}\label{sec:PS_Mesh}

We propose a representation of a mesh as a collection of cells which are represented likewise. collection of faces plus their respective attributes. This representation accounts for orthogonal cartesian meshes. Those are generated using the number of cells in each axis or direction, the thickness and top for each cell. Nevertheless, the thickness is only needed for the number of cells defined in every axis, because we work with regular meshes. Therefore the rest of the cells will have the same thickness.
\\
The top of the mesh is required for the first XY plane, and needs to be filled with the depths of each cell in that plane, the rest of the cells are calculated using the depth of the first plane.

The representation stated for Mesh only accounts for orthogonal cartesian meshes, which can be generated with information about number of cells in each axis, their thickness and tops. A (The information above is inserted by a) Geomodeler with defines the mesh by inserting for each axis the number of cells and the thickness for cells in that direction. Once 
\ref{fig:Mesh}.\\
\begin{figure}[h]
	\centering%
	\includegraphics[width=0.9\linewidth]{Kap4/Mesh.pdf}%
	\caption{Mesh definition.} \label{fig:Mesh}
\end{figure}

\subsection{Rock}\label{sec:PS_Rock}
\ref{fig:Rock}.\\
\begin{figure}[h]
	\centering%
	\includegraphics[width=0.9\linewidth]{Kap4/Rock.pdf}%
	\caption{Fluid Characterization.} \label{fig:Rock}
\end{figure}
\subsection{Phase}\label{sec:PS_Phase}

\ref{fig:Fluid}.\\
\begin{figure}[h]
	\centering%
	\includegraphics[width=0.9\linewidth]{Kap4/Fluid.pdf}%
	\caption{Fluid Characterization.} \label{fig:Fluid}
\end{figure}

\subsection{Inter-phase interaction}\label{sec:PS_Interphase}

%\subsection{Component}\label{sec:PS_Component} % This could be changed to Chemical

\subsection{Equilibrium Relation}\label{sec:PS_Equilibrium}

\subsection{Well}\label{sec:PS_Well}

%includefigure{Graphical Representation of subroutine}

%includegraphic{Code Translation}

%Se deben incluir tantos cap\'{\i}tulos como se requieran; sin embargo, se recomienda que la tesis  o trabajo de investigaci\'{o}n tenga un m\'{\i}nimo 3 cap\'{\i}tulos y m\'{a}ximo de 6 cap\'{\i}tulos (incluyendo las conclusiones).\\
\chapter{Validación y Resultados}\label{cap:Validacion}
%Se deben incluir tantos cap\'{\i}tulos como se requieran; sin embargo, se recomienda que la tesis  o trabajo de investigaci\'{o}n tenga un m\'{\i}nimo 3 cap\'{\i}tulos y m\'{a}ximo de 6 cap\'{\i}tulos (incluyendo las conclusiones).\\

%\renewcommand{\tablename}{\textbf{Código}}

En este capítulo se pretende validar el modelo ejecutable que se realiza a partir del esquema preconceptual. Con éste propósito, se plantea un caso que se toma de \cite{jamal2006petroleum}. El cual, consiste en una simulación de un yacimiento lineal (1D) de cinco celdas con un solo fluido y un pozo productor en la cuarta celda, como se muestra en la Figura \ref{fig:Abou-Kassem} \citep{jamal2006petroleum}. Con este, se espera validar el transporte de un fluido y las pérdidas de presión por pozos.\\

El caso de estudio de \cite{jamal2006petroleum} consta de dos procesos principales, uno de pérdidas a caudal fijo, y el otro, cuando la celda perforada para el pozo alcanza la ``presión de abandono''. En este, se cambia la condición operativa del pozo para mantener la presión fija a esa presión de abandono. En la Figura \ref{fig:ConstantQ} se reportan los resultados de la simulación durante la etapa de producción a caudal constante, y en la Figura \ref{fig:ConstantP} se reportan los resultados de la simulación durante la etapa de producción a presión constante.
\begin{figure}[h]
	\centering
	\includegraphics[width=0.9\textwidth]{Fig/casoasis.pdf}
	\caption{Representación gráfica del caso proporcionado por \cite{jamal2006petroleum}.}
	\label{fig:Abou-Kassem}
\end{figure}



\begin{figure}[h!]
	\centering
\begin{tikzpicture}
\begin{axis}[
title={Perdidas de presión a caudal fijo. Simulado vs Reportado},
xlabel={Celda},
ylabel={Presión (psi)},
xmin=0, xmax=6,
ymin=1500, ymax=3000,
legend pos=outer north east ,
ymajorgrids=true,
xmajorgrids=true,
%semilogxaxis=true,
grid style=dashed,
]

\addplot[color=blue]
coordinates{
	(1,2933.8141602)
	(2,2923.5889812)
	(3,2908.882128)
	(4,2896.698936)
	(5,2899.4836656)
	
};
\addlegendentry{5 días, simulado.}

\addplot[dashed, color=red]
coordinates{
	(1,2605.9267536)
	(2,2595.745086)
	(3,2581.1107518)
	(4,2569.0000788)
	(5,2571.741297)
	
};
\addlegendentry{25 días, simulado.}

\addplot[dotted,color=black]
coordinates{
	(1,2203.1127162)
	(2,2193.0180714)
	(3,2178.5287752)
	(4,2166.5196288)
	(5,2169.2463432)
	
};
\addlegendentry{50 días, simulado.}

\addplot[dashdotted,color=purple]
coordinates{
	(1,1729.7812032)
	(2,1719.8025888)
	(3,1705.4728344)
	(4,1693.6087260)
	(5,1696.3064328)
};
\addlegendentry{80 días, simulado.}

\addplot[only marks, mark options={draw=blue,fill=blue,}]
coordinates{
	(1,2936.80)
	(2,2928.38)
	(3,2915.68)
	(4,2904.88)
	(5,2908.18)
	
	
};
\addlegendentry{5 días, reportado.}
\addplot[only marks, mark=square*, mark options={draw=red,fill=red,}]
coordinates{
	(1,2630.34)
	(2,2620.06)
	(3,2605.28)
	(4,2593.04)
	(5,2595.81)
	
	
};
\addlegendentry{25 días, reportado.}
\addplot[only marks, mark=diamond*, mark options={draw=black,fill=black,}]
coordinates{
	(1,2243.97)
	(2,2233.68)
	(3,2218.9)
	(4,2206.66)
	(5,2209.43)
	
	
};
\addlegendentry{50 días, reportado.}
\addplot[only marks, mark=triangle*, mark options={draw=purple,fill=purple,}]
coordinates{
	
	(1,1780.3100)
	(2,1770.0200)
	(3,1755.2400)
	(4,1743.0000)
	(5,1745.7700)
	
	
};
\addlegendentry{80 días, reportado.}
\end{axis}
\end{tikzpicture}
\caption{Comparativo datos simulados contra caso de estudio de \cite{jamal2006petroleum}. Caudal constante.}
\label{fig:ConstantQ}
\end{figure}{}

Las condiciones iniciales de presión para el yacimiento son de $3000$ $psia$ para todas las celdas y el pozo se establece a una condición operativa de caudal fijo a $400$ $STB/D$. Se puede ver como en los primeros 80 días, va decayendo la presión hasta alcanzar la presión de abandono. Se encuentran diferencias entre los resultados simulados y los reportados por \cite{jamal2006petroleum} menores a $50$ $psia$. Éstas se deben, a que las ecuaciones que se implementan en el modelo ejecutable, son diferentes a las que se presentan en \cite{jamal2006petroleum}.

\begin{figure}[h!]
	\centering
	\begin{tikzpicture}
	\begin{axis}[
	title={Pérdidas de presión a presión fija. Simulado vs Reportado},
	xlabel={Celda},
	ylabel={Presión (psi)},
	xmin=0, xmax=6,
	ymin=1500, ymax=1550,
	legend pos=outer north east ,
	ymajorgrids=true,
	xmajorgrids=true,
	%semilogxaxis=true,
	grid style=dashed,
	]
	
	\addplot[color=blue]
	coordinates{
		(1,	1526.960064 )
		(2,	1522.7829696)
		(3,	1517.169999)
		(4,	1512.6593172)
		(5,	1513.1959578)
	};
	\addlegendentry{100 días, simulado.}
	
	\addplot[color=green]
	coordinates{
		(1,	1509.9326028)
		(2,	1508.4097038)
		(3,	1506.3356604)
		(4,	1504.6822272)
		(5,	1504.8852804)
	};
	\addlegendentry{105 días, simulado.}
	
	\addplot[color=black]
	coordinates{
		(1,	1503.6524574)
		(2,	1503.101313)
		(3,	1502.3326116)
		(4,	1501.723452)
		(5,	1501.795971)
	};
	\addlegendentry{110 días, simulado.}
	
	\addplot[color=purple]
	coordinates{
		(1,	1501.3463532)
		(2,	1501.1433)
		(3,	1500.853224)
		(4,	1500.635667)
		(5,	1500.6501708)
	};
	\addlegendentry{115 días, simulado.}
	
	\addplot[color=pink]
	coordinates{
		(1,	1500.490629)
		(2,	1500.41811)
		(3,	1500.3165834)
		(4,	1500.2295606)
		(5,	1500.2295606)
	};
	\addlegendentry{120 días, simulado.}
	
	
	\addplot[]
	coordinates{
		(1,	1500.1715454)
		(2,	1500.1425378)
		(3,	1500.1135302)
		(4,	1500.0845226)
		(5,	1500.0845226)
	};
	\addlegendentry{125 días, simulado.}
	
	\addplot[color=orange]
	coordinates{
		(1,	1500.055515)
		(2,	1500.0410112)
		(3,	1500.0265074)
		(4,	1500.0265074)
		(5,	1500.0265074)
	};
	\addlegendentry{130 días, simulado.}
	
	
	\addplot[red]
	coordinates{
		(1,	1500.0120036)
		(2,	1500.0120036)
		(3,	1500.0120036)
		(4,	1499.9974998)
		(5,	1499.9974998)
	};
	\addlegendentry{135 días, simulado.}
	
	
	\addplot[only marks]
	coordinates{
		(1,	1524.61)
		(2,	1520)
		(3,	1514)
		(4,	1509 )
		(5,	1510)
		
	};
	\addlegendentry{100 días, reportado. }
	
	\addplot[only marks]
	coordinates{
		(1,	1510.35)
		(2,	1508.46)
		(3,	1505.91)
		(4,	1503.88)
		(5,	1504.09)
		
	};
	\addlegendentry{105 días, reportado.}
	
	\addplot[only marks]
	coordinates{
		(1,	1504.34)
		(2,	1503.54)
		(3,	1502.47)
		(4,	1501.61)
		(5,	1501.7)
	};
	\addlegendentry{110 días, reportado. }
	
	\addplot[only marks]
	coordinates{
		(1,	1501.82)
		(2,	1501.48)
		(3,	1501.03)
		(4,	1500.68)
		(5,	1500.71)
	};
	\addlegendentry{115 días, reportado.}
	
	\addplot[only marks]
	coordinates{
		(1,	1500.76)
		(2,	1500.62)
		(3,	1500.43)
		(4,	1500.28)
		(5,	1500.3)
	};
	\addlegendentry{120 días, reportado.}
	
	
	\addplot[only marks]
	coordinates{
		(1,	1500.32)
		(2,	1500.26)
		(3,	1500.18)
		(4,	1500.12)
		(5,	1500.12)
	};
	\addlegendentry{125 días, reportado.}
	
	\addplot[only marks]
	coordinates{
		(1,	1500.13)
		(2,	1500.11)
		(3,	1500.08)
		(4,	1500.05)
		(5,	1500.05)
	};
	\addlegendentry{130 días, reportado.}
	
	
	\addplot[only marks]
	coordinates{
		(1	,1500.06)
		(2	,1500.05)
		(3	,1500.03)
		(4	,1500.02)
		(5	,1500.02)
	};
	\addlegendentry{135 días, reportado.}
	
	\end{axis}
	\end{tikzpicture}
	\caption{Comparativo datos simulados contra caso de estudio de \cite{jamal2006petroleum}. Presión constante.}
	\label{fig:ConstantP}
\end{figure}{}

En la etapa de producción a presión constante se logra observar como la presión del sistema decae hasta que se estabiliza completamente a la presión de abandono. En esta etapa se siguen observando diferencias de alrededor de $10$ $psia$. Con este comportamiento se logra validar el modelo ejecutable para la simulación de casos de producción por flujo natural. Para ejecutar casos en los que se involucren más fenómenos físicos, se requiere un control numérico adicional que no se contempla en el desarrollo de esta Tesis de Maestría.

\chapter{Conclusiones y trabajo futuro}\label{cap:Conclusiones}
\section{Conclusiones}
La concordancia de los resultados con el caso de la literatura indican que el modelo ejecutable desarrollado tiene validez, el desarrollo del esquema preconceptual ejecutable y su respectiva traducción a C++ es una clara muestra de la consistencia que se obtiene al trazar los conceptos y los procesos en el código. La representación en el EP es acorde con la física y la conceptualización de los modelos matemáticos.

\section{Trabajo futuro}

A partir de esta Tesis de Maestría se identifican varias líneas de investigación por desarrollar:

\begin{itemize}
	\item Representar la fenomenología de diferentes químicos usando el modelo ejecutable definido.
	\item Modelar un simulador composicional de flujo basado en esquemas preconceptuales.
	\item Explorar diferentes esquemas de solución numérica para la simulación de procesos de recobro mejorado en yacimientos de petróleo.
	\item Identificar patrones de diseño en el modelo ejecutable basado en esquemas preconceptuales definido.
\end{itemize}

%Se presentan como una serie de aspectos que se podr\'{\i}an realizar en un futuro para emprender investigaciones similares o fortalecer la investigaci\'{o}n realizada. Deben contemplar las perspectivas de la investigaci\'{o}n, las cuales son sugerencias, proyecciones o alternativas que se presentan para modificar, cambiar o incidir sobre una situaci\'{o}n espec\'{\i}fica o una problem\'{a}tica encontrada. Pueden presentarse como un texto con caracter\'{\i}sticas argumentativas, resultado de una reflexi\'{o}n acerca de la tesis o trabajo de investigaci\'{o}n.\\
\begin{appendix}
\chapter{Anexo: Discretización de las Ecuaciones del BOM}\label{AnexoA}
En este apartado se presenta la discretización de las Ecuaciones del BOM \ref{ec:aceite}, \ref{ec:gas}, \ref{ec:agua} que se presentan en la subsección \ref{subsec:BOM}. Recordando:
\begin{align*}
\text{aceite: }&\frac{\partial}{\partial t} \left[ \phi \left( \frac{S_{o}}{B_{o}} + \frac{R_{v} S_{g}}{B_{g}} \right) \right]
- \nabla \cdot \left( \frac{1}{B_{o}} \vec{u_{o}} + \frac{R_{v}}{B_{g}} \vec{u_{g}} \right) - \tilde{q}_{o}=0  \\
\text{gas: }&\frac{\partial}{\partial t} \left[ \phi \left( \frac{S_{g}}{B_{g}} + \frac{R_{s} S_{o}}{B_{o}} \right) \right]
- \nabla \cdot \left( \frac{1}{B_{g}} \vec{u_{g}} + \frac{R_{s}}{B_{o}} \vec{u_{o}} \right) - \tilde{q}_{g} = 0 \\
\text{agua: }&\frac{\partial}{\partial t} \left[\phi \left( \frac{S_{w}}{B_{w}} \right) \right] - \nabla \cdot \left( \frac{1}{B_{w}} \vec{u_{w}} \right) - \tilde{q}_{w} = 0 
\end{align*}
Primero, se integran las ecuaciones sobre un intervalo de tiempo $\left[t, t+\Delta t\right]$, y una celda de control $\Omega_{i}$. Se toma como ejemplo la ecuación de conservación para el aceite \ref{ec:aceite}. Las ecuaciones para el gas y agua (\ref{ec:gas}, \ref{ec:agua}) se discretizan de manera análoga.
\begin{align*}
	\int_{t}^{t+\Delta t}\int_{\Omega_{i}}\frac{\partial}{\partial t} \left[ \phi \left( \frac{S_{o}}{B_{o}} + \frac{R_{v} S_{g}}{B_{g}} \right) \right]dVdt
	- \int_{t}^{t+\Delta t}\int_{\Omega_{i}}\nabla \cdot \left( \frac{1}{B_{o}} \vec{u_{o}} + \frac{R_{v}}{B_{g}} \vec{u_{g}} \right)dVdt \\
	- \int_{t}^{t+\Delta t}\int_{\Omega_{i}}\tilde{q}_{o}dVdt=0 
\end{align*}
Luego, se desglosan las ecuaciones en sus términos acumulativos, de flujo, y de fuentes y sumideros, debido a que cada uno de estos términos requiere un tratamiento distinto.
\begin{align*}
\underbrace{\int_{t}^{t+\Delta t}\int_{\Omega_{i}}\frac{\partial}{\partial t} \left[ \phi \left( \frac{S_{o}}{B_{o}} + \frac{R_{v} S_{g}}{B_{g}} \right) \right]dVdt}_{\text{Término de acumulación}}
- \underbrace{\int_{t}^{t+\Delta t}\int_{\Omega_{i}}\nabla \cdot \left( \frac{1}{B_{o}} \vec{u_{o}} + \frac{R_{v}}{B_{g}} \vec{u_{g}} \right)dVdt}_\text{Término de flujo} \\
- \underbrace{\int_{t}^{t+\Delta t}\int_{\Omega_{i}}\tilde{q}_{o}dVdt}_{\text{Término de fuentes y sumideros}}=0 
\end{align*}
Empezando por el término acumulativo, se sigue que, por el teorema de Fubini que
\begin{align*}
	\int_{t}^{t+\Delta t}\left(\int_{\Omega_{i}}\frac{\partial}{\partial t} \left[ \phi \left( \frac{S_{o}}{B_{o}} + \frac{R_{v} S_{g}}{B_{g}} \right) \right]dV\right)dt = \int_{\Omega_{i}}\left(\int_{t}^{t+\Delta t}\frac{\partial}{\partial t} \left[ \phi \left( \frac{S_{o}}{B_{o}} + \frac{R_{v} S_{g}}{B_{g}} \right) \right]dt\right)dV 
\end{align*}
Por teorema fundamental del cálculo
\begin{align*}
 \int_{\Omega_{i}}\left(\int_{t}^{t+\Delta t}\frac{\partial}{\partial t} \left[ \phi \left( \frac{S_{o}}{B_{o}} + \frac{R_{v} S_{g}}{B_{g}} \right) \right]dt\right)dV = \int_{\Omega_{i}}\left[ \phi \left( \frac{S_{o}}{B_{o}} + \frac{R_{v} S_{g}}{B_{g}} \right) \right]^{t+\Delta t}_{t}dV
\end{align*}
Considerando el cambio de la acumulación en un intervalo de tiempo como una constante para una celda de control $\Omega_{i}$
\begin{align*}
	\int_{\Omega_{i}}\left[ \phi \left( \frac{S_{o}}{B_{o}} + \frac{R_{v} S_{g}}{B_{g}} \right) \right]^{t+\Delta t}_{t}dV = \left[ \phi_{i} \left( \frac{S_{o,i}}{B_{o,i}} + \frac{R_{v,i} S_{g,i}}{B_{g,i}} \right) \right]^{t+\Delta t}_{t}\int_{\Omega_{i}}dV \\= |\Omega_{i}|\left[ \phi_{i} \left( \frac{S_{o,i}}{B_{o,i}} + \frac{R_{v,i} S_{g,i}}{B_{g,i}} \right) \right]^{t+\Delta t}_{t}
\end{align*}
Donde $|\Omega_{i}|$ es el volumen para una celda $i$. Continuando con la discretización del término de flujo, se discretiza primero el tiempo usando un esquema \textbf{implícito}, luego
\begin{align*}
\int_{t}^{t+\Delta t}\int_{\Omega_{i}}\nabla \cdot \left( \frac{1}{B_{o}} \vec{u_{o}} + \frac{R_{v}}{B_{g}} \vec{u_{g}} \right)dVdt = \Delta t \left[\int_{\Omega_{i}}\nabla \cdot \left( \frac{1}{B_{o}} \vec{u_{o}} + \frac{R_{v}}{B_{g}} \vec{u_{g}} \right)dV\right]^{t+\Delta t}
\end{align*}
Posteriormente, se asume que la función vectorial de velocidad $\vec{u_{f}}$ para cualquier fluido $f$ tiene derivadas de primer orden continuas. Luego, como la celda tiene una superficie cerrada, se aplica el teorema de la divergencia Gauss.
\begin{align*}
	\int_{\Omega_{i}}\nabla \cdot \left( \frac{1}{B_{o}} \vec{u_{o}} + \frac{R_{v}}{B_{g}} \vec{u_{g}} \right)dV = \int_{\partial \Omega_{i}}\left( \frac{1}{B_{o}} \vec{u_{o}} + \frac{R_{v}}{B_{g}} \vec{u_{g}} \right)\cdot \partial \vec{S_{i}}
\end{align*}
Usando celdas rectangulares, la integral sobre la frontera de una celda es la suma del término de flujo que se evalúa en cada cara, así
\begin{align*}
 \int_{\partial \Omega_{i}}\left( \frac{1}{B_{o}} \vec{u_{o}} + \frac{R_{v}}{B_{g}} \vec{u_{g}} \right)\cdot \partial \vec{S_{i}} = \sum_{c \in S_{i}}\int_{c}\left( \frac{1}{B_{o}} \vec{u_{o}} + \frac{R_{v}}{B_{g}} \vec{u}_{g} \right)\cdot \partial \vec{S_{c}}
\end{align*}
El valor del término de flujo en cada cara se aproxima usando el teorema del valor medio, luego el producto punto se evalua como el valor de la función en la cara
\begin{align*}
	\sum_{c \in S_{i}}\int_{c}\left( \frac{1}{B_{o}} \vec{u_{o}} + \frac{R_{v}}{B_{g}} \vec{u_{g}} \right)\cdot \partial \vec{S_{c}} \approx \sum_{c \in S_{i}} A_{c} \left( \frac{1}{B_{o,c}} \vec{u_{o,c}} + \frac{R_{v,c}}{B_{g,c}} \vec{u_{g,c}} \right)
\end{align*}
Aplicando las velocidades Darcy en la ecuación, se tiene
\begin{align*}
	\sum_{c \in S_{i}} A_{c} \left( \frac{1}{B_{o,c}} \vec{u_{o,c}} + \frac{R_{v,c}}{B_{g,c}} \vec{u_{g,c}} \right) \cdot \vec{n_{c}} = \sum_{c \in S_{i}} A_{c} \left( \frac{1}{B_{o,c}} \frac{\mathbb{K}_{c}k_{ro,c}}{\mu_{o,c}} \nabla \Phi_{o,c} + \frac{R_{v,c}}{B_{g,c}} \frac{\mathbb{K}_{c}k_{rg,c}}{\mu_{g,c}} \nabla \Phi_{g,c} \right)
\end{align*}
Dónde el gradiente de potencial $\nabla \Phi_{f,c}$, para cualquier fluido $f$ se aproxima como:
\begin{align*}
	\nabla \Phi_{f,c} \approx \frac{\Delta \Phi_{f,c}}{\Delta l_{c}} \approx \frac{\Phi_{f,i}-\Phi_{f,j}}{\Delta l_{c}}
\end{align*}
Donde $c$ es la cara que conecta la celda $i$ con la celda $j$, y el delta de longitud en la cara $\Delta l_{c}$ se calcula como:
\begin{align}
	\label{ec:DeltaCara}\Delta l_{c} = \frac{\Delta l_{i} + \Delta l_{j}}{2}
\end{align}
Cabe notar que, los deltas de longitud dependen de la dirección del vector normal $\vec{n}$. Por último, el término de fuentes y sumideros se aproxima de la misma manera que el termino del flujo, usando el teorema de valor medio. Así
\begin{align*}
	\int_{t}^{t+\Delta t}\int_{\Omega_{i}}\tilde{q}_{o}dVdt \approx \int_{t}^{t+\Delta t}\tilde{q}_{o,i} |\Omega_{i}|dt
	\approx \left[\tilde{q}_{o,i}\right]^{t+\Delta t} |\Omega_{i}|\Delta t
\end{align*}
tomando los tiempos $t+\Delta t$ como tiempo futuro o $n+1$ y los tiempo $t$ como tiempo pasado $n$, se obtiene la discretización de la ecuación de conservación del aceite (\ref{ec:aceiteDiscretizacion}) como:
\begin{align*}
	|\Omega_{i}|\left[ \phi_{i} \left( \frac{S_{o,i}}{B_{o,i}} + \frac{R_{v,i} S_{g,i}}{B_{g,i}} \right) \right]^{n+1}_{n} - \Delta t \sum_{c \in S_{i}} A_{c} \left[ \frac{1}{B_{o,c}} \frac{\mathbb{K}_{c}k_{ro,c}}{\mu_{o,c}} \nabla \Phi_{o,c} + \frac{R_{v,c}}{B_{g,c}} \frac{\mathbb{K}_{c}k_{rg,c}}{\mu_{g,c}} \nabla \Phi_{g,c} \right]^{n+1} \\- \left[\tilde{q}_{o,i}\right]^{n+1} |\Omega_{i}|\Delta t = 0
\end{align*}
Se define el término de transmisividad, en el cual se considera un promedio armónico que se utiliza para estimar el término $\mathbb{K}_{c}$ y se involucra la Ecuación \ref{ec:DeltaCara}, de la siguiente forma:
\begin{align*}
	&T_{o,c} = \left(\frac{2}{(\Delta l_{i}/A_{c}K_{l,i})+(\Delta l_{j}/A_{c}K_{l,j})}\right)\frac{k_{ro,c}}{\mu_{o,c}B_{o,c}}\\
	&T_{g,c} = \left(\frac{2}{(\Delta l_{i}/A_{c}K_{l,i})+(\Delta l_{j}/A_{c}K_{l,j})}\right)\frac{k_{rg,c}}{\mu_{g,c}B_{g,c}}
\end{align*}
Luego la ecuación de conservación que se discretiza queda:
\begin{align*}
|\Omega_{i}|\left[ \phi_{i} \left( \frac{S_{o,i}}{B_{o,i}} + \frac{R_{v,i} S_{g,i}}{B_{g,i}} \right) \right]^{n+1}_{n} - \Delta t \sum_{c \in S_{i}} \left[ T_{o,c} \Delta \Phi_{o,c} + R_{v,c} T_{g,c} \nabla \Phi_{g,c} \right]^{n+1} - \left[\tilde{q}_{o,i}\right]^{n+1} |\Omega_{i}|\Delta t = 0
\end{align*}

Considerando $\left[\tilde{q}_{o,i}\right]^{n+1} |\Omega_{i}|$ como un flujo volumétrico $\left[Q_{o,i}\right]^{n+1}$ e involucrando la evaluación al tiempo $n+1$ en cada uno de los términos, se tiene:
\begin{align*}
|\Omega_{i}|\left[ \phi_{i} \left( \frac{S_{o,i}}{B_{o,i}} + \frac{R_{v,i} S_{g,i}}{B_{g,i}} \right) \right]^{n+1}_{n} - \Delta t \sum_{c \in S_{i}} \left[ T_{o,c}^{n+1} \Delta \Phi_{o,c}^{n+1} + R_{v,c}^{n+1} T_{g,c}^{n+1} \Delta \Phi_{g,c}^{n+1} \right] - Q_{o,i}^{n+1} \Delta t
\end{align*}
Finalmente, dividiendo por $\Delta t$ la ecuación se obtiene:
\begin{align*}
\frac{|\Omega_{i}|}{\Delta t}\left[ \phi_{i} \left( \frac{S_{o,i}}{B_{o,i}} + \frac{R_{v,i} S_{g,i}}{B_{g,i}} \right) \right]^{n+1}_{n} - \sum_{c \in S_{i}} \left[ T_{o,c}^{n+1} \Delta \Phi_{o,c}^{n+1} + R_{v,c}^{n+1} T_{g,c}^{n+1} \Delta \Phi_{g,c}^{n+1} \right] - Q_{o,i}^{n+1} = 0
\end{align*}
La discretización de las ecuaciones de conservación para el gas y el agua (\ref{ec:gas}, \ref{ec:agua}) se obtienen de manera análoga a la del aceite.
%\chapter{Anexo: Traducciones a código de la representación}
%En este apartado se muestran porciones del modelo ejecutable que se programa en C++ \citep{ISO:2017:IIIa}. Se elije este lenguaje porque la \textit{Stardard Template Library} (STL) tiene facilidades que permiten hacer la consistencia del código con el esquema preconceptual explícita. Se emula una base de datos a partir de arreglos globales que almacenan cada concepto, y, las claves foráneas que se generan entre conceptos se traducen como punteros. \\
%
%A continuación se muestra, a modo de ejemplo, las traducciones de algunos conceptos, funciones, o porciones del EP a código C++ para verificar la consistencia del código con el modelo ejecutable. Primero, se definen los elementos que conforman las condiciones iniciales, y, la base de datos que se emula, es decir, los arreglos que contienen los conceptos que se instancian en la ejecución de las relaciones dinámicas del EP. Posteriormente, se muestran las traducciones a código de dos conceptos clase, ``Malla'' y ``Roca'', y la traducción de la relación dinámica ``Petrofísico caracteriza Roca''. Luego, se explica la traducción del cálculo del residual en el evento ``Presión del Fluido Varía''.\\
%
%Las condiciones iniciales, en las que se definen variables globales y constantes, se programan en un \textit{namespace} ``Global'' que agrupa todas las definiciones, tal como aparecen en la figura \ref{fig:EjInitialConditions}. El código para las condiciones iniciales se expone en la tabla \ref{tab:InitialConditions}.\\
%
%\begin{table}[h]
%	\centering
%	\begin{tabular}{cc}
%		\parbox[c]{10em}{
%			\begin{tabular}[c]{@{}c@{}}\includegraphics[width=3in]{Fig/EjInitialConditions.pdf}\\ Ver figura \ref{fig:EjInitialConditions}.\end{tabular}
%			%
%			%\caption[Elementos para la representación de Software Científico.]{Elementos para la representación de Software Científico. \citep{JCalle,norena2018det}.} \label{fig:RockTranslation}
%		}
%		&
%		\begin{tiny}
%			\begin{lstlisting}
%			namespace Initial_Conditions{
%			
%			std::string timestamp="";
%			double mytime=0;
%			double simulationtime = 86400;
%			double timedelta=1;
%			int wells_quantity=0;
%			int term=0;
%			int fluids_quantity=0;
%			int stencil[2] = {-1,1};
%			int equilibrium_relations_quantity=0;
%			int interfluid_interactions_quantity=0;
%			int cells_number=0;
%			int changing_wells=0;
%			
%			};
%			
%			\end{lstlisting}
%		\end{tiny}
%	\end{tabular}
%	\label{tab:InitialConditions}
%	\caption[Traducción a código de las condiciones iniciales.]{Traducción a código de las condiciones iniciales. Los autores.}
%\end{table}
%
%En el código \ref{tab:bd} se muestra los conceptos que se almacenan en arreglos, estos se iteran en diferentes funciones del código. Es importante notar que, en las precondiciones del EP se establece que sólo se define una única malla y sólo se caracteriza una única roca. Por tanto, en la base de datos que se emula, estos dos conceptos se acceden por medio de un puntero único. También se destaca que, todos los objetos que se instancian de los conceptos clase, se acceden a partir de punteros. Todos los punteros se fundamentan en la STL, que se encarga de hacer la respectiva gestión del uso de la memoria.\\
%
%\begin{table}
%	\begin{tabular}{c}
%		\begin{tiny}
%			\begin{lstlisting}
%			
%			std::vector<std::shared_ptr<Equation_Base>> equations =
%			std::vector<std::shared_ptr<Equation_Base>>();
%			
%			std::vector<std::shared_ptr<Fluid>> characterized_fluids =
%			std::vector<std::shared_ptr<Fluid>>();
%			
%			std::vector<std::unique_ptr<Equilibrium_Relation>> added_equilibrium_relations =
%			std::vector<std::unique_ptr<Equilibrium_Relation>>();
%			
%			std::vector<std::unique_ptr<Interfluid_Interaction>> added_interfluid_interactions =
%			std::vector<std::unique_ptr<Interfluid_Interaction>>();
%			
%			std::vector<std::shared_ptr<Well>> perforated_wells =
%			std::vector<std::shared_ptr<Well>>();
%			
%			std::unique_ptr<Mesh> mymesh;
%			std::unique_ptr<Rock> myrock;
%			
%			\end{lstlisting}
%		\end{tiny}
%	\end{tabular}
%	\label{tab:bd}
%	\caption[Arreglos que conforman la base de datos que se emula.]{Arreglos que conforman la base de datos que se emula. Los autores.}
%\end{table}
%
%En el código \ref{tab:MeshCode} se puede observar que, la conceptualización de la malla coincide con el código que se genera para la misma. La malla se propone como un conjunto de celdas. Adicionalmente, se tienen elementos como el espesor que dependen de la cantidad de celdas en cada dirección. Las celdas también se iteran, pero su arreglo correspondiente se almacena dentro del concepto malla, tal como se propone en la sección \ref{subsec:PS_Mesh}.\\
%
%\begin{table}[h]
%	\centering
%	\begin{tabular}{cc}
%		\parbox[c]{5em}{
%			\begin{tabular}[c]{@{}c@{}}\includegraphics[width=2in]{Fig/Mesh.pdf}\\ Ver figura \ref{fig:Mesh}.\end{tabular}
%			%
%			%\caption[Elementos para la representación de Software Científico.]{Elementos para la representación de Software Científico. \citep{JCalle,norena2018det}.} \label{fig:RockTranslation}
%		}
%		&
%		\begin{tiny}
%			\begin{lstlisting}
%			
%			class Mesh{
%			
%			private:
%			
%			using Cells_t = std::vector<std::shared_ptr<Cell>>;
%			int _dimension;
%			std::vector<int> _cell_number = std::vector<int>(3);
%			int _cell_total;
%			std::vector<std::vector<double>> _thickness;
%			std::vector<std::vector<double>> _top;
%			int _defined=0;
%			Cells_t _cells;
%			
%			public:
%			
%			using Cell_iterator = Cells_t::iterator;
%			using Cell_const_iterator = Cells_t::const_iterator;
%			
%			Mesh();
%			int getCellTotal();
%			void define();
%			void defineFromFile(std::ifstream& mesh_reader);
%			void appear(const std::string& _timestamp, const int stencil[2]);
%			
%			int listCell(int posx, int posy, int posz);
%			int listCell(std::vector<int> _Numeration);
%			
%			const std::shared_ptr<Cell>& cell(const int index) const 
%			{return _cells[index];};
%			
%			const double& thickness(const int axis, const int spacing) const 
%			{return _thickness[axis][spacing];};
%			
%			Cell_iterator begin() {return _cells.begin();};
%			Cell_iterator end()   {return _cells.end();};
%			
%			Cell_const_iterator begin()  const {return _cells.begin();};
%			Cell_const_iterator end()    const {return _cells.end();};
%			Cell_const_iterator cbegin() const {return _cells.cbegin();};
%			Cell_const_iterator cend()   const {return _cells.cend();};
%			
%			
%			};
%			
%			\end{lstlisting}
%		\end{tiny}
%	\end{tabular}
%	\label{tab:MeshCode}
%	\caption[Traducción a código del concepto Malla.]{Traducción a código del concepto Malla. Los autores.}
%\end{table}
%
%
%
%\begin{table}[h]
%	\centering
%	\begin{tabular}{cc}
%		\parbox[c]{1em}{
%			\begin{tabular}[c]{@{}c@{}}\includegraphics[width=1in]{Fig/Rock.pdf}\\ Ver figura \ref{fig:Rock}.\end{tabular}
%			%
%			%\caption[Elementos para la representación de Software Científico.]{Elementos para la representación de Software Científico. \citep{JCalle,norena2018det}.} \label{fig:RockTranslation}
%		}
%		&
%		\begin{tiny}
%			\begin{lstlisting}
%			class Rock{
%			private:
%			
%			double _reference_pressure;
%			double _compressibility;
%			std::vector<std::vector<std::vector<double>>> _absolute_permeability;   
%			std::vector<std::vector<double>> _porosity;
%			
%			public:
%			
%			Rock(){};
%			void characterize(const int& cells_number);
%			void characterizeFromFile(std::ifstream& rock_reader,
%			const int& cells_number);
%			void porosity(const int term, const int cell_index,
%			const double pressure);
%			
%			void updateProperties(const int& term);
%			
%			const double& porosity (const int term, const int cells_number) const {
%			return _porosity[term][cells_number];
%			};
%			
%			const std::vector<double>& absolutePermeability
%			(const int term, const int cells_number) const {
%			return _absolute_permeability[term][cells_number];
%			};
%			};
%			
%			\end{lstlisting}
%		\end{tiny}
%	\end{tabular}
%	\label{tab:RockCode}
%	\caption[Traducción a código del concepto Roca.]{Traducción a código del concepto Roca. Los autores.}
%\end{table}
%
%\begin{table}[h]
%	\centering
%	\begin{tabular}{cc}
%		\parbox[c]{5em}{
%			\begin{tabular}[c]{@{}c@{}}\includegraphics[width=1.5in]{Fig/Rock.pdf}\\ Ver figura \ref{fig:Rock}.\end{tabular}
%			%
%			%\caption[Elementos para la representación de Software Científico.]{Elementos para la representación de Software Científico. \citep{JCalle,norena2018det}.} \label{fig:RockTranslation}
%		}
%		&
%		\begin{tiny}
%			\begin{lstlisting}
%			void Rock::characterize(const int& cells_number){
%			
%			std::ostringstream ss = std::ostringstream();
%			const std::string axisnames[3]={"x", "y", "z"};
%			
%			_absolute_permeability = std::vector<std::vector<std::vector<double>>>
%			(1,std::vector<std::vector<double>>(cells_number,std::vector<double>(3)));
%			_porosity              = std::vector<std::vector<double>>(1,std::vector<double>(cells_number));
%			
%			myRead(std::string("Please insert rock compressibility [1/Pa]"), _compressibility,
%			std::string("Please insert a valid input"));
%			
%			myRead(std::string("Please insert reference pressure [Pa]"), _reference_pressure,
%			std::string("Please insert a valid input"));
%			
%			for(int cellindex=0; cellindex<cells_number; ++cellindex){
%			
%			ss << "Please insert initial porosity for the "<< cellindex+1 << " cell [-]";
%			myRead(ss.str(), _porosity[0][cellindex], std::string("Please insert a valid input"));
%			ss.str("");
%			ss.clear();
%			
%			
%			};
%			for(int cellindex=0; cellindex<cells_number; ++cellindex){
%			for(int direction=0; direction<3;++direction){
%			ss << "Please insert initial absolute permeability for the "<< cellindex+1
%			<< " cell in direction " << axisnames[direction] << " [m2]";
%			myRead(ss.str(), _absolute_permeability[0][cellindex][direction],
%			std::string("Please insert a valid input"));
%			ss.str("");
%			ss.clear();
%			};
%			};
%			};
%			
%			\end{lstlisting}
%		\end{tiny}
%	\end{tabular}
%	\label{tab:RockCharacterizeCode}
%	\caption[Traducción a código de la relación dinámica ``Petrofísico caracteriza Roca''.]{Traducción a código de la relación dinámica ``Petrofísico caracteriza Roca''. Los autores.}
%\end{table}
%
%\begin{table}[h]
%	\centering
%	\begin{tabular}{cc}
%		\parbox[c]{7em}{
%			\begin{tabular}[c]{@{}c@{}}\includegraphics[width=2.5in]{Fig/Residual.pdf}\\ Ver figura \ref{fig:Residual}.\end{tabular}
%			%
%			%\caption[Elementos para la representación de Software Científico.]{Elementos para la representación de Software Científico. \citep{JCalle,norena2018det}.} \label{fig:RockTranslation}
%		}
%		&
%		\begin{tiny}
%			\begin{lstlisting}
%			//Residual calculation
%			residual_selector = 0;
%			
%			for(auto equation : equations){
%			
%			if(equation->status()){
%			
%			if(equation->type() == typeid(Well).name()){
%			
%			constexpr auto residual_type = "well";
%			auto residual_well = std::dynamic_pointer_cast<Well,Equation_Base>(equation);
%			
%			row = locate(residual_type, residual_selector, residual_well->index());
%			_residual(row) = calculateWellResidual(term, residual_well);
%			
%			}else{
%			
%			constexpr auto residual_type = "fluid";
%			auto residual_fluid = std::dynamic_pointer_cast<Fluid,Equation_Base>(equation);
%			
%			for(auto cell = mesh.begin(); cell != mesh.end(); ++cell){
%			
%			cell_index = (*cell)->index();
%			
%			row = locate(residual_type, residual_selector, cell_index);
%			
%			_residual(row) = 
%			calculateResidual(term,*residual_fluid, mesh, *cell, rock, wells);
%			
%			};
%			};
%			
%			++residual_selector;
%			
%			};
%			};
%			
%			\end{lstlisting}
%		\end{tiny}
%	\end{tabular}
%	\label{tab:ResidualCode}
%	\caption[Traducción a código del cálculo de residual en el evento ``Presión del Fluido Varía''.]{Traducción a código del cálculo de residual en el evento ``Presión del Fluido Varía''. Los autores.}
%\end{table}

%\chapter{Anexo: Pirobos todos}\label{AnexoSizas}
%\newpage
%\begin{small}
%	\begin{lstlisting}
%	double calculateAccumulation
%	    (const int& term, Fluid& fluid, 
%	        const std::shared_ptr<Cell>& cell, Rock& rock){
%	
%	double past_contribution=0;
%	double current_contribution=0;
%	
%	const int cell_index = cell->index();
%	
%	for(auto& equilibrium_relation : added_equilibrium_relations ){
%	
%	if(equilibrium_relation->receiverFluid()->index() == fluid.index()){
%	
%	const auto contributor = equilibrium_relation->contributorFluid();
%	
%	double past_coef = 
%	    equilibrium_relation->partitionCoefficient(term-1,cell_index);
%	
%	past_contribution = past_contribution +
%	past_coef * (rock.porosity(term-1,cell_index) * 
%	contributor->saturation(term-1,cell_index)
%	/ contributor->volumetricFactor(term-1,cell_index));
%	
%	double curr_coef = 
%	    equilibrium_relation->partitionCoefficient(term,cell_index);
%	
%	current_contribution = current_contribution +
%	curr_coef * (rock.porosity(term,cell_index) * 
%	contributor->saturation(term,cell_index)
%	/ contributor->volumetricFactor(term,cell_index));
%	
%	};
%	};
%	
%	double accumulation = (cell->volume()/Initial_Conditions::timedelta) *
%	(((rock.porosity(term,cell_index)*fluid.saturation(term,cell_index)
%	/fluid.volumetricFactor(term,cell_index)) + current_contribution)-
%	((rock.porosity(term-1,cell_index)*fluid.saturation(term-1,cell_index)
%	/fluid.volumetricFactor(term-1,cell_index)) + past_contribution));
%	
%	return accumulation;
%	};
%	\end{lstlisting}
%\end{small}

%A final del documento es opcional incluir \'{\i}ndices o glosarios. \'{E}stos son listas detalladas y especializadas de los t\'{e}rminos, nombres, autores, temas, etc., que aparecen en el mismo. Sirven para facilitar su localizaci\'{o}n en el texto. Los \'{\i}ndices pueden ser alfab\'{e}ticos, cronol\'{o}gicos, num\'{e}ricos, anal\'{\i}ticos, entre otros. Luego de cada palabra, t\'{e}rmino, etc., se pone coma y el n\'{u}mero de la p\'{a}gina donde aparece esta informaci\'{o}n.\\

%\chapter{Anexo: Nombrar el anexo C de acuerdo con su contenido}
%MANEJO DE LA BIBLIOGRAF\'{I}A: la bibliograf\'{\i}a es la relaci\'{o}n de las fuentes documentales consultadas por el investigador para sustentar sus trabajos. Su inclusi\'{o}n es obligatoria en todo trabajo de investigaci\'{o}n. Cada referencia bibliogr\'{a}fica se inicia contra el margen izquierdo.\\
%
%La NTC 5613 establece los requisitos para la presentaci\'{o}n de referencias bibliogr\'{a}ficas citas y notas de pie de p\'{a}gina. Sin embargo, se tiene la libertad de usar cualquier norma bibliogr\'{a}fica de acuerdo con lo acostumbrado por cada disciplina del conocimiento. En esta medida es necesario que la norma seleccionada se aplique con rigurosidad.\\
%
%Es necesario tener en cuenta que la norma ISO 690:1987 (en Espa\~{n}a, UNE 50-104-94) es el marco internacional que da las pautas m\'{\i}nimas para las citas bibliogr\'{a}ficas de documentos impresos y publicados. A continuaci\'{o}n se lista algunas instituciones que brindan par\'{a}metros para el manejo de las referencias bibliogr\'{a}ficas:\\
%
%\begin{center}
%\centering%
%\begin{tabular}{|p {7.5 cm}|p {7.5 cm}|}\hline
%\arr{Instituci\'{o}n}&Disciplina de aplicaci\'{o}n\\\hline%
%Modern Language Association (MLA)&Literatura, artes y humanidades\\\hline%
%American Psychological Association (APA)&Ambito de la salud (psicolog\'{\i}a, medicina) y en general en todas las ciencias sociales\\\hline
%Universidad de Chicago/Turabian &Periodismo, historia y humanidades.\\\hline
%AMA (Asociaci\'{o}n M\'{e}dica de los Estados Unidos)&Ambito de la salud (psicolog\'{\i}a, medicina)\\\hline
%Vancouver &Todas las disciplinas\\\hline
%Council of Science Editors (CSE)&En la actualidad abarca diversas ciencias\\\hline
%National Library of Medicine (NLM) (Biblioteca Nacional de Medicina)&En el \'{a}mbito m\'{e}dico y, por extensi\'{o}n, en ciencias.\\\hline
%Harvard System of Referencing Guide &Todas las disciplinas\\\hline
%JabRef y KBibTeX &Todas las disciplinas\\\hline
%\end{tabular}
%\end{center}
%
%Para incluir las referencias dentro del texto y realizar lista de la bibliograf\'{\i}a en la respectiva secci\'{o}n, puede utilizar las herramientas que Latex suministra o, revisar el instructivo desarrollado por el Sistema de Bibliotecas de la Universidad Nacional de Colombia\footnote{Ver: www.sinab.unal.edu.co}, disponible en la secci\'{o}n "Servicios", opci\'{o}n "Tr\'{a}mites" y enlace "Entrega de tesis".

%\chapter{Sizas Tikz}
%{\color{red} \LARGE Ojo! Esto no es un anexo, es solo para pruebas}
%\begin{figure}[h!]
%	\begin{tikzpicture}
%	\begin{axis}[
%	title={Presión por Bloque},
%	xlabel={Bloque},
%	ylabel={Presión (psi)},
%	xmin=0, xmax=6,
%	ymin=1500, ymax=1550,
%	legend pos=outer north east ,
%	ymajorgrids=true,
%	xmajorgrids=true,
%	%semilogxaxis=true,
%	grid style=dashed,
%	]
%	
%	\addplot[color=blue]
%	coordinates{
%		(1,	1526.960064 )
%		(2,	1522.7829696)
%		(3,	1517.169999)
%		(4,	1512.6593172)
%		(5,	1513.1959578)
%	};
%	\addlegendentry{5 días. Autor.}
%	
%	\addplot[color=green]
%	coordinates{
%		(1,	1509.9326028)
%		(2,	1508.4097038)
%		(3,	1506.3356604)
%		(4,	1504.6822272)
%		(5,	1504.8852804)
%	};
%	\addlegendentry{10 días. Autor.}
%	
%	\addplot[color=black]
%	coordinates{
%		(1,	1503.6524574)
%		(2,	1503.101313)
%		(3,	1502.3326116)
%		(4,	1501.723452)
%		(5,	1501.795971)
%	};
%	\addlegendentry{15 días. Autor.}
%	
%	\addplot[color=purple]
%	coordinates{
%		(1,	1501.3463532)
%		(2,	1501.1433)
%		(3,	1500.853224)
%		(4,	1500.635667)
%		(5,	1500.6501708)
%	};
%	\addlegendentry{20 días. Autor.}
%	
%	\addplot[color=pink]
%	coordinates{
%		(1,	1500.490629)
%		(2,	1500.41811)
%		(3,	1500.3165834)
%		(4,	1500.2295606)
%		(5,	1500.2295606)
%	};
%	\addlegendentry{25 días. Autor.}
%	
%	
%	\addplot[]
%	coordinates{
%		(1,	1500.1715454)
%		(2,	1500.1425378)
%		(3,	1500.1135302)
%		(4,	1500.0845226)
%		(5,	1500.0845226)
%	};
%	\addlegendentry{30 días. Autor.}
%	
%	\addplot[color=orange]
%	coordinates{
%		(1,	1500.055515)
%		(2,	1500.0410112)
%		(3,	1500.0265074)
%		(4,	1500.0265074)
%		(5,	1500.0265074)
%	};
%	\addlegendentry{35 días. Autor.}
%	
%	
%	\addplot[red]
%	coordinates{
%		(1,	1500.0120036)
%		(2,	1500.0120036)
%		(3,	1500.0120036)
%		(4,	1499.9974998)
%		(5,	1499.9974998)
%	};
%	\addlegendentry{40 días. Autor.}
%	
%	
%	\addplot[only marks]
%	coordinates{
%		(1,	1524.61)
%		(2,	1520)
%		(3,	1514)
%		(4,	1509 )
%		(5,	1510)
%		
%	};
%	\addlegendentry{5 días. }
%	
%	\addplot[only marks]
%	coordinates{
%		(1,	1510.35)
%		(2,	1508.46)
%		(3,	1505.91)
%		(4,	1503.88)
%		(5,	1504.09)
%		
%	};
%	\addlegendentry{10 días.}
%	
%	\addplot[only marks]
%	coordinates{
%		(1,	1504.34)
%		(2,	1503.54)
%		(3,	1502.47)
%		(4,	1501.61)
%		(5,	1501.7)
%	};
%	\addlegendentry{15 días. }
%	
%	\addplot[only marks]
%	coordinates{
%		(1,	1501.82)
%		(2,	1501.48)
%		(3,	1501.03)
%		(4,	1500.68)
%		(5,	1500.71)
%	};
%	\addlegendentry{20 días.}
%	
%	\addplot[only marks]
%	coordinates{
%		(1,	1500.76)
%		(2,	1500.62)
%		(3,	1500.43)
%		(4,	1500.28)
%		(5,	1500.3)
%	};
%	\addlegendentry{25 días.}
%	
%	
%	\addplot[only marks]
%	coordinates{
%		(1,	1500.32)
%		(2,	1500.26)
%		(3,	1500.18)
%		(4,	1500.12)
%		(5,	1500.12)
%	};
%	\addlegendentry{30 días.}
%	
%	\addplot[only marks]
%	coordinates{
%		(1,	1500.13)
%		(2,	1500.11)
%		(3,	1500.08)
%		(4,	1500.05)
%		(5,	1500.05)
%	};
%	\addlegendentry{35 días.}
%	
%	
%	\addplot[only marks]
%	coordinates{
%		(1	,1500.06)
%		(2	,1500.05)
%		(3	,1500.03)
%		(4	,1500.02)
%		(5	,1500.02)
%	};
%	\addlegendentry{40 días.}
%	
%	\end{axis}
%	\end{tikzpicture}
%	\caption{Comparativo datos simulados contra caso de estudio de \cite{jamal2006petroleum}.}
%	\label{cora}
%\end{figure}{}
%
%
%\begin{figure}[h!]
%	\begin{tikzpicture}
%	\begin{axis}[
%	title={Datos simulados por el Autor contra caso de estudio.},
%	xlabel={Bloque},
%	ylabel={Presión (psi)},
%	xmin=0, xmax=6,
%	ymin=1500, ymax=3000,
%	legend pos=outer north east ,
%	ymajorgrids=true,
%	xmajorgrids=true,
%	%semilogxaxis=true,
%	grid style=dashed,
%	]
%	
%	\addplot[]
%	coordinates{
%		(1,2933.8141602)	
%		(2,2923.5889812)	
%		(3,2908.882128)	
%		(4,2896.698936)	
%		(5,2899.4836656)
%		
%	};
%	\addlegendentry{5 días.}
%	
%	\addplot[]
%	coordinates{
%		(1,2605.9267536)	
%		(2,2595.745086)	
%		(3,2581.1107518)	
%		(4,2569.0000788)	
%		(5,2571.741297)
%		
%	};
%	\addlegendentry{25 días.}
%	
%	\addplot[]
%	coordinates{
%		(1,2203.1127162)	
%		(2,2193.0180714)	
%		(3,2178.5287752)	
%		(4,2166.5196288)	
%		(5,2169.2463432)
%		
%	};
%	\addlegendentry{50 días.}
%	
%	\addplot[]
%	coordinates{
%		(1,1729.7812032)	
%		(2,1719.8025888)	
%		(3,1705.4728344)	
%		(4,1693.6087260)
%		(5,1696.3064328)
%		
%	};
%	\addlegendentry{80 días.}
%	\end{axis}
%	\end{tikzpicture}
%	\caption{gg izi}
%	\label{siiiizas}
%\end{figure}{}
%
%\begin{figure}[h!]
%	\begin{tikzpicture}
%	\begin{axis}[
%	title={Datos simulados por el Autor contra caso de estudio.},
%	xlabel={Bloque},
%	ylabel={Presión (psi)},
%	xmin=0, xmax=6,
%	ymin=1500, ymax=3000,
%	legend pos=outer north east ,
%	ymajorgrids=true,
%	xmajorgrids=true,
%	%semilogxaxis=true,
%	grid style=dashed,
%	]
%	
%	\addplot[]
%	coordinates{
%		(1,2933.8141602)
%		(2,2923.5889812)
%		(3,2908.882128)
%		(4,2896.698936)
%		(5,2899.4836656)
%		
%	};
%	\addlegendentry{5 días.}
%	
%	\addplot[]
%	coordinates{
%		(1,2605.9267536)
%		(2,2595.745086)
%		(3,2581.1107518)
%		(4,2569.0000788)
%		(5,2571.741297)
%		
%	};
%	\addlegendentry{25 días.}
%	
%	\addplot[]
%	coordinates{
%		(1,2203.1127162)
%		(2,2193.0180714)
%		(3,2178.5287752)
%		(4,2166.5196288)
%		(5,2169.2463432)
%		
%	};
%	\addlegendentry{50 días.}
%	
%	\addplot[]
%	coordinates{
%		(1,1729.7812032)
%		(2,1719.8025888)
%		(3,1705.4728344)
%		(4,1693.6087260)
%		(5,1696.3064328)
%	};
%	\addlegendentry{80 días.}
%	
%	\addplot[only marks]
%	coordinates{
%		(1,2936.80)
%		(2,2928.38)
%		(3,2915.68)
%		(4,2904.88)
%		(5,2908.18)
%		
%		
%	};
%	\addlegendentry{días, reportado.}
%	\addplot[only marks]
%	coordinates{
%		(1,2630.34)
%		(2,2620.06)
%		(3,2605.28)
%		(4,2593.04)
%		(5,2595.81)
%		
%		
%	};
%	\addlegendentry{días, reportado.}
%	\addplot[only marks]
%	coordinates{
%		(1,2243.97)
%		(2,2233.68)
%		(3,2218.9)
%		(4,2206.66)
%		(5,2209.43)
%		
%		
%	};
%	\addlegendentry{días, reportado.}
%	\addplot[only marks]
%	coordinates{
%		
%		(1,1780.3100)
%		(2,1770.0200)
%		(3,1755.2400)
%		(4,1743.0000)
%		(5,1745.7700)
%		
%		
%	};
%	\addlegendentry{días, reportado.}
%	\end{axis}
%	\end{tikzpicture}
%	\caption{gg}
%	\label{recora}
%\end{figure}{}

\end{appendix}
\addcontentsline{toc}{chapter}{\numberline{}Bibliograf\'{\i}a}
\bibliographystyle{apa}
\bibliography{BibliMSc}
\end{document}