%\newpage
%\setcounter{page}{1}
\begin{center}
\begin{figure}
\centering%
\epsfig{file=HojaTitulo/EscudoUN,scale=1}%
\end{figure}
\thispagestyle{empty} \vspace*{0.0cm} \textbf{\LARGE
Un modelo ejecutable para la simulación multi-f\'{\i}sica de procesos de recobro mejorado en yacimientos de petr\'{o}leo basado en esquemas preconceptuales}\\[5.3cm]
\Large\textbf{Steven Velásquez Chancí}\\[5.3cm]
\small Universidad Nacional de Colombia\\
Facultad de Minas, Departamento de Ciencias de la Computación y la Decisión\\
Medellín, Colombia\\
2019\\
\end{center}

\newpage
\begin{center}
\thispagestyle{empty} \vspace*{0cm} \textbf{\LARGE
Un modelo ejecutable para la simulación multi-f\'{\i}sica de procesos de recobro mejorado en yacimientos de petr\'{o}leo basado en esquemas preconceptuales}\\[2.0cm]
\Large\textbf{Steven Velásquez Chancí}\\[2.0cm]
\small Tesis de investigación presentada como requisito parcial para optar al
t\'{\i}tulo de:\\
\textbf{Magíster en Ingeniería de Sistemas}\\[2.0cm]
Director:\\
Ph.D. Juan Manuel Mejía Cárdenas\\
Co-Director:\\
Ph.D. Carlos Mario Zapata Jaramillo\\[2.0cm]
L\'{\i}nea de Investigaci\'{o}n:\\
Ingeniería de Software\\
Grupos de Investigaci\'{o}n:\\
Dinámicas de Flujo y Transporte en medios porosos\\
Lenguajes Computacionales\\[2.0cm]
Universidad Nacional de Colombia\\
Facultad de Minas, Departamento de Ciencias de la Computación y la Decisión\\
Medellín, Colombia\\
2019\\
\end{center}

\newpage{\pagestyle{empty}\cleardoublepage}

\newpage
\thispagestyle{empty} \textbf{}\normalsize
\\\\\\%
~\\[4.0cm]

\begin{flushright}
\begin{minipage}{8cm}
    \noindent
        \small
        ~\\[2.0cm]
        A mi familia, amigos y a todos los que\\
        estuvieron pendientes de este proceso.\\
        ¡Se logró!
        
        %La preocupaci\'{o}n por el hombre y su destino siempre debe ser el
        %inter\'{e}s primordial de todo esfuerzo t\'{e}cnico. Nunca olvides esto
        %entre tus diagramas y ecuaciones.\\\\
        %Albert Einstein\\
\end{minipage}
\end{flushright}

\chapter*{Agradecimientos}
\addcontentsline{toc}{chapter}{\numberline{}Agradecimientos}
Agradezco a mi Director Juan Manuel Mejía Cardenas por su supervisión y confianza durante este proceso, a mi Codirector Carlos Mario Zapata Jaramillo por su guía y por saberme centrar cuando lo he requerido, a Juan David Valencia Londoño, cuya constante disposición para solucionar mis dudas me ayudaron a llevar un avance continuo, a Felipe Ospina Montoya, cuyo apoyo ha sido invaluable, y a Paola Andrea Nore\~{n}a Cardona, por su fé, apoyo, asesoría y acompañamiento constante durante el desarrollo de esta Tesis.\\

Finalmente, agradezco al Departamiento de Ciencias de la Computación y la Decisión por financiar mis estudios por medio de la beca de Facultad y a la alianza formada por el Departamento Administrativo de Ciencia, Tecnología e Innovación (COLCIENCIAS), la Agencia Nacional de Hidrocarburos (ANH) y la Universidad Nacional de Colombia por el financiamiento del ``Plan Nacional para el Potenciamiento de la Tecnología CEOR con Gas Mejorado Químicamente'' bajo el acuerdo 273-2017, dentro del cual se enmarca la investigación asociada a esta Tesis de Maestría.

\newpage{\pagestyle{empty}\cleardoublepage}

\chapter*{Resumen}
\addcontentsline{toc}{chapter}{\numberline{}Resumen}

La simulación de procesos de recobro mejorado se rige por las leyes en las que se describe el transporte de fluidos en medios porosos. Existen múltiples propuestas de elaboración de \textit{frameworks} y simuladores para procesos de recobro mejorado. Sin embargo, carecen de trazabilidad de conceptos, procesos y de representación de eventos que surgen de la física. En los Esquemas Preconceptuales (EP) se incluye toda la estructura de un dominio de aplicación y, también se pueden representar los procesos que se dan en tal dominio. Estos, aportan cohesión, consistencia y trazabilidad entre conceptos y procesos. Por lo que, en esta Tesis de Maestría se propone un modelo ejecutable para la simulación de procesos de recobro mejorado basado en esquemas preconceptuales. El modelo ejecutable se valida con un caso de estudio. Los resultados del modelo ejecutable se ajustan a los datos reportados en la literatura. El modelo ejecutable propuesto permite trazar consistentemente los conceptos, procesos y eventos presentes en la simulación de procesos de recobro mejorado.\\


\textbf{\small Palabras clave: Modelo Ejecutable, Esquemas Preconceptuales, Simulación de Yacimientos de Petróleo, Procesos de Recobro Mejorado, Representación Basada en Eventos}.\\

%El resumen es una presentaci\'{o}n abreviada y precisa (la NTC 1486 de 2008 recomienda revisar la norma ISO 214 de 1976). Se debe usar una extensi\'{o}n m\'{a}xima de 12 renglones. Se recomienda que este resumen sea anal\'{\i}tico, es decir, que sea completo, con informaci\'{o}n cuantitativa y cualitativa, generalmente incluyendo los siguientes aspectos: objetivos, dise\~{n}o, lugar y circunstancias, pacientes (u objetivo del estudio), intervenci\'{o}n, mediciones y principales resultados, y conclusiones. Al final del resumen se deben usar palabras claves tomadas del texto (m\'{\i}nimo 3 y m\'{a}ximo 7 palabras), las cuales permiten la recuperaci\'{o}n de la informaci\'{o}n.\\


%\textbf{\small Palabras clave: (m\'{a}ximo 10 palabras, preferiblemente seleccionadas de las listas internacionales que permitan el indizado cruzado)}.\\

%A continuaci\'{o}n se presentan algunos ejemplos de tesauros que se pueden consultar para asignar las palabras clave, seg\'{u}n el \'{a}rea tem\'{a}tica:\\

%\textbf{Artes}: AAT: Art y Architecture Thesaurus.

%\textbf{Ciencias agropecuarias}: 1) Agrovoc: Multilingual Agricultural Thesaurus - F.A.O. y 2)GEMET: General Multilingual Environmental Thesaurus.

%\textbf{Ciencias sociales y humanas}: 1) Tesauro de la UNESCO y 2) Population Multilingual Thesaurus.

%\textbf{Ciencia y tecnolog\'{\i}a}: 1) Astronomy Thesaurus Index. 2) Life Sciences Thesaurus, 3) Subject Vocabulary, Chemical Abstracts Service y 4) InterWATER: Tesauro de IRC - Centro Internacional de Agua Potable y Saneamiento.

%\textbf{Tecnolog\'{\i}as y ciencias m\'{e}dicas}: 1) MeSH: Medical Subject Headings (National Library of Medicine's USA) y 2) DECS: Descriptores en ciencias de la Salud (Biblioteca Regional de Medicina BIREME-OPS).

%\textbf{Multidisciplinarias}: 1) LEMB - Listas de Encabezamientos de Materia y 2) LCSH- Library of Congress Subject Headings.\\

%Tambi\'{e}n se pueden encontrar listas de temas y palabras claves, consultando las distintas bases de datos disponibles a trav\'{e}s del Portal del Sistema Nacional de Bibliotecas\footnote{ver: www.sinab.unal.edu.co}, en la secci\'{o}n "Recursos bibliogr\'{a}ficos" opci\'{o}n "Bases de datos".\\

\chapter*{Abstract}
\addcontentsline{toc}{chapter}{\numberline{}Abstract}
Enhanced oil recovery (EOR) processes simulation is governed by mass conservation laws. In such laws, flow, accumulation, sources and sinks phenomena in porous media are described. Multiple proposals for framework and simulation elaboration have been defined. However, they lack concepts and processes tracing and event representation for physical phenomena. Preconceptual Schemas (PS) are used for including the complete structure of an application domain and representing processes emerging in such a domain. Cohesion, consistency, and tracing between concepts and processes is obtained by using PS. In this MSc. Thesis an executable model for enhanced oil recovery processes simulation based on preconceptual schemas is proposed. The executable model is validated by running a study case. The results are in accordance with data reported in the literature. The proposed executable model allows for tracing consistently the concepts, processes, and events, which are present in EOR processes simulation. \\

\textbf{\small Palabras clave: Executable Models, Preconceptual Schemas, Oil Reservoir Simulation, Enhanced Oil Recovery Processes, Event based Representation}.\\

%Es el mismo resumen pero traducido al ingl\'{e}s. Se debe usar una extensi\'{o}n m\'{a}xima de 12 renglones. Al final del Abstract se deben traducir las anteriores palabras claves tomadas del texto (m\'{\i}nimo 3 y m\'{a}ximo 7 palabras), llamadas keywords. Es posible incluir el resumen en otro idioma diferente al espa\~{n}ol o al ingl\'{e}s, si se considera como importante dentro del tema tratado en la investigaci\'{o}n, por ejemplo: un trabajo dedicado a problemas ling\"{u}\'{\i}sticos del mandar\'{\i}n seguramente estar\'{\i}a mejor con un resumen en mandar\'{\i}n.\\[2.0cm]
%\textbf{\small Keywords: palabras clave en ingl\'{e}s(m\'{a}ximo 10 palabras, preferiblemente seleccionadas de las listas internacionales que permitan el indizado cruzado)}\\