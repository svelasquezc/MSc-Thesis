%\chapter{Cap\'{\i}tulo 3}
\chapter{Solution Proposal}

In this section we propose two elements as an extension for Preconceptual Schemas(PS) which aid the understanding of diverse elements in the oil reservoir simulation domain. In addition, we present further description of the concepts stated in the theoretical framework, with their respective representation in the elaborated PS and their code translation.
\\
This section is structured as follows: in section 4.1 we present the added elements to PS and their usage in our represented domain. In section 4.2 we propose the representation of structural and dynamical behavior of each developed concept in the theoretical framework using PS.

\section{Added elements to Preconceptual Schemas}
\subsection{Analyst defined subroutines}
Analyst defined subroutines are analyst defined functions as proposed by (ref Calle) without the return argument. They use global elements and parameters of the subroutine definition. They are defined for re-using dynamical behavior elements which appear more than once in the PS. Names of both subroutines and functions must differ from operators predefined in the PS. Graphic symbol used for subroutines is the same as used for operators and functions. In figure \# we present graphical representation of analyst defined subroutines.

\section{PS Representation of Enhanced Oil Recovery Processes}
In this section we propose a PS representation for enhanced oil recovery processes, in the graphical representation the authors couple a black oil model discretized using finite volumes method with the theoretical framework developed the previous chapter. We mapped each term in the resultant equations to their respective concepts 

%includefigure{Graphical Representation of subroutine}

%includegraphic{Code Translation}

%Se deben incluir tantos cap\'{\i}tulos como se requieran; sin embargo, se recomienda que la tesis  o trabajo de investigaci\'{o}n tenga un m\'{\i}nimo 3 cap\'{\i}tulos y m\'{a}ximo de 6 cap\'{\i}tulos (incluyendo las conclusiones).\\