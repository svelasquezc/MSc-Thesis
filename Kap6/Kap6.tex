\chapter{Conclusiones y trabajo futuro}\label{cap:Conclusiones}
\section{Conclusiones}


En esta Tesis de Maestría se establecieron los fenómenos de flujo o transporte, transferencia y de superficie en los procesos de recobro mejorado (EOR) para yacimientos de petróleo. A partir de tales fenómenos, se conceptualizaron los elementos del sistema, o dominio, y las interacciones físicas, y, químicas entre fluidos y agentes externos que se presentan en los procesos EOR. La estrategia de solución que se diseñó, permite acoplar diferentes mecanismos físicos y químicos de manera general, solucionando las ecuaciones algebraicas y diferenciales en las que se describen estos mecanismos. Con esta estrategia de solución, se construyó una simulación basada en un esquema preconceptual que hace las veces de modelo ejecutable y, que se logró validar con un caso de estudio en la literatura.\\

Si bien existen múltiples representaciones para la simulación de procesos de recobro mejorado, y se han realizado diferentes simulaciones a partir de sus modelos matemáticos, se logró evidenciar que muchas de estas carecen de trazabilidad entre conceptos y procesos que se ejecutan la simulación, induciendo a una falta de formalidad en el desarrollo de simuladores para procesos EOR. Este modelo ejecutable puede constituir una base para desarrollar simulaciones de procesos EOR, debido a que conserva la trazabilidad de los conceptos presentes y muestra el paso a paso de la ejecución de los procesos. Éste, también presenta elementos de generalización que permiten abarcar otros fenómenos. Además, la concordancia de los resultados con el caso de la literatura indica que el modelo ejecutable desarrollado tiene validez. Más aún, es posible, por medio de los eventos del EP, representar la aplicación de otros campos en el dominio físico, como, por ejemplo, los de temperatura y esfuerzos.\\

El desarrollo del esquema preconceptual ejecutable y su respectiva traducción a C++ es una clara muestra de la consistencia que se obtiene al trazar los conceptos y los procesos en el código y, la representación en el EP es acorde con la física y la conceptualización de los modelos matemáticos que se establecen para una simulación. Un elemento importante en el modelo ejecutable es la capacidad de encapsular e iterar sobre conceptos, generando independencia entre los mismos. Esto provee la flexibilidad suficiente para generalizar los modelos que se utilizan en la simulación de yacimientos de petróleo, y cambiar especificaciones sin modificar el modelo ejecutable.

\section{Trabajo futuro}

A partir de esta Tesis de Maestría se identifican varias líneas de investigación por desarrollar:

\begin{itemize}
	\item Representar la fenomenología de diferentes químicos usando el modelo ejecutable definido.
	\item Modelar un simulador composicional de flujo basado en esquemas preconceptuales.
	\item Explorar diferentes esquemas de solución numérica para la simulación de procesos de recobro mejorado en yacimientos de petróleo.
	\item Identificar patrones de diseño en el modelo ejecutable basado en esquemas preconceptuales definido.
\end{itemize}

\chapter*{Productos Académicos}

De esta Tesis de Maestría se realizaron los siguientes módulos de software registrados:

\begin{itemize}
	\item DFT-UPGRADING TOOL - Número de registro: 13-76-119.
	\item DFT-GAS TOOL - Número de registro: 13-76-118.
	\item DFT-CEOR TOOL - Número de registro: 13-76-108.
\end{itemize}

%Se presentan como una serie de aspectos que se podr\'{\i}an realizar en un futuro para emprender investigaciones similares o fortalecer la investigaci\'{o}n realizada. Deben contemplar las perspectivas de la investigaci\'{o}n, las cuales son sugerencias, proyecciones o alternativas que se presentan para modificar, cambiar o incidir sobre una situaci\'{o}n espec\'{\i}fica o una problem\'{a}tica encontrada. Pueden presentarse como un texto con caracter\'{\i}sticas argumentativas, resultado de una reflexi\'{o}n acerca de la tesis o trabajo de investigaci\'{o}n.\\