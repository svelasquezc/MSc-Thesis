\chapter{Conclusiones y trabajo futuro}\label{cap:Conclusiones}
\section{Conclusiones}
La concordancia de los resultados con el caso de la literatura indican que el modelo ejecutable desarrollado tiene validez, el desarrollo del esquema preconceptual ejecutable y su respectiva traducción a C++ es una clara muestra de la consistencia que se obtiene al trazar los conceptos y los procesos en el código. La representación en el EP es acorde con la física y la conceptualización de los modelos matemáticos.

\section{Trabajo futuro}

A partir de esta Tesis de Maestría se identifican varias líneas de investigación por desarrollar:

\begin{itemize}
	\item Representar la fenomenología de diferentes químicos usando el modelo ejecutable definido.
	\item Modelar un simulador composicional de flujo basado en esquemas preconceptuales.
	\item Explorar diferentes esquemas de solución numérica para la simulación de procesos de recobro mejorado en yacimientos de petróleo.
	\item Identificar patrones de diseño en el modelo ejecutable basado en esquemas preconceptuales definido.
\end{itemize}

%Se presentan como una serie de aspectos que se podr\'{\i}an realizar en un futuro para emprender investigaciones similares o fortalecer la investigaci\'{o}n realizada. Deben contemplar las perspectivas de la investigaci\'{o}n, las cuales son sugerencias, proyecciones o alternativas que se presentan para modificar, cambiar o incidir sobre una situaci\'{o}n espec\'{\i}fica o una problem\'{a}tica encontrada. Pueden presentarse como un texto con caracter\'{\i}sticas argumentativas, resultado de una reflexi\'{o}n acerca de la tesis o trabajo de investigaci\'{o}n.\\