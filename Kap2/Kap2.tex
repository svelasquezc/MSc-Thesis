%\chapter{Cap\'{\i}tulo 1}
%Los cap\'{\i}tulos son las principales divisiones del documento. En estos, se desarrolla el tema del documento. Cada cap\'{\i}tulo debe corresponder a uno de los temas o aspectos tratados en el documento y por tanto debe llevar un t\'{\i}tulo que indique el contenido del cap\'{\i}tulo.\\
%
%Los t\'{\i}tulos de los cap\'{\i}tulos deben ser concertados entre el alumno y el director de la tesis  o trabajo de investigaci\'{o}n, teniendo en cuenta los lineamientos que cada unidad acad\'{e}mica brinda. As\'{\i} por ejemplo, en algunas facultades se especifica que cada cap\'{\i}tulo debe corresponder a un art\'{\i}culo cient\'{\i}fico, de tal manera que se pueda publicar posteriormente en una revista.\\
%
%\section{Subt\'{\i}tulos nivel 2}
%Toda divisi\'{o}n o cap\'{\i}tulo, a su vez, puede subdividirse en otros niveles y s\'{o}lo se enumera hasta el tercer nivel. Los t\'{\i}tulos de segundo nivel se escriben con min\'{u}scula al margen izquierdo y sin punto final, est\'{a}n separados del texto o contenido por un interlineado posterior de 10 puntos y anterior de 20 puntos (tal y como se presenta en la plantilla).\\
%
%\subsection{Subt\'{\i}tulos nivel 3}
%De la cuarta subdivisi\'{o}n en adelante, cada nueva divisi\'{o}n o \'{\i}tem puede ser se\~{n}alada con vi\~{n}etas, conservando el mismo estilo de \'{e}sta, a lo largo de todo el documento.\\
%
%Las subdivisiones, las vi\~{n}etas y sus textos acompa\~{n}antes deben presentarse sin sangr\'{\i}a y justificados.\\
%
%\begin{itemize}
%\item En caso que sea necesario utilizar vi\~{n}etas, use este formato (vi\~{n}etas cuadradas).
%\end{itemize}

\chapter{Marco teórico}
%
%\section{Black Oil Model}
\section{Procesos de Recobro Mejorado}

\section{Simulación de Yacimientos de Petróleo}
\subsection{Modelo Black Oil Extendido}

El \textit{Black Oil} es un modelo de transporte simultáneo de tres fluidos en el que se asume que los hidrocarburos se distribuyen en un gas y un aceite en barril a condiciones de presión y temperatura estándar \citep{chen2007reservoir}. Este modelo considera que puede haber una transferencia de masa en equilibrio desde aceite al gas, lo que se denomina ``Gas disuelto''. Adicionalmente, en el modelo \textit{Black Oil} extendido, se considera una transferencia de masa desde el gas al aceite, lo que se denomina ``aceite volatilizado''.

\begin{align}
\label{ec:aceite}
\text{aceite: }&\frac{\partial}{\partial t} \left[ \phi \left( \frac{S_{o}}{B_{o}} + \frac{R_{v} S_{g}}{B_{g}} \right) \right]
+ \nabla \cdot \left( \frac{1}{B_{o}} \vec{v_{o}} + \frac{R_{v}}{B_{g}} \vec{v_{g}} \right) + Q_{o} = 0 \\
\label{ec:gas}
\text{gas: }&\frac{\partial}{\partial t} \left[ \phi \left( \frac{S_{g}}{B_{g}} + \frac{R_{s} S_{o}}{B_{o}} \right) \right]
+ \nabla \cdot \left( \frac{1}{B_{g}} \vec{v_{g}} + \frac{R_{s}}{B_{o}} \vec{v_{o}} \right) + Q_{g} = 0\\
\label{ec:agua}
\text{agua: }&\frac{\partial}{\partial t} \left[\phi \left( \frac{S_{w}}{B_{w}} \right) \right] + \nabla \cdot \left( \frac{1}{B_{w}} \vec{v_{w}} \right) + Q_{w} = 0
\end{align}
donde $\vec{v_{p}}$ corresponde la ley de Darcy para el fluido $p = \left\lbrace o:\text{ aceite}, g:\text{ gas}, w:\text{ agua} \right\rbrace $:
\begin{equation}
\vec{v_{p}}=\frac{\mathbb{K} kr_{p}}{\mu_{p} } \nabla{\Phi_{p,f}}
\end{equation}
\subsection{Problema de Valores Iniciales}
%
\subsection{Modelamiento de Pozos}
%
\subsection{Modelamiento del Químico}
%
\subsection{Discretización}
%Algebraic Equations

\subsection{Método de Newton-Raphson}
%

\section{Esquemas Preconceptuales}
Los Esquemas Preconceptuales (EP) son representaciones intermedias entre el lenguaje natural y un esquema conceptual o un lenguaje formal \citep{zapata2007phd}. Sirven para establecer un punto común de entendimiento entre un interesado y un analista.
\subsection{Elementos Principales}

\subsection{Representación Basada en Eventos}