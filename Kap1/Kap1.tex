% !TeX spellcheck = es
\chapter{Introducci\'{o}n}
%En la introducci\'{o}n, el autor presenta y se\~{n}ala la importancia, el origen (los antecedentes te\'{o}ricos y pr\'{a}cticos), los objetivos, los alcances, las limitaciones, la metodolog\'{\i}a empleada, el significado que el estudio tiene en el avance del campo respectivo y su aplicaci\'{o}n en el \'{a}rea investigada. No debe confundirse con el resumen y se recomienda que la introducci\'{o}n tenga una extensi\'{o}n de m\'{\i}nimo 2 p\'{a}ginas y m\'{a}ximo de 4 p\'{a}ginas.\\
%
%La presente plantilla maneja una familia de fuentes utilizada generalmente en LaTeX, conocida como Computer Modern, espec\'{\i}ficamente LMRomanM para el texto de los p\'{a}rrafos y CMU Sans Serif para los t\'{\i}tulos y subt\'{\i}tulos. Sin embargo, es posible sugerir otras fuentes tales como Garomond, Calibri, Cambria, Arial o Times New Roman, que por claridad y forma, son adecuadas para la edici\'{o}n de textos acad\'{e}micos.\\
%
%La presente plantilla tiene en cuenta aspectos importantes de la Norma T\'{e}cnica Colombiana - NTC 1486, con el fin que sea usada para la presentaci\'{o}n final de las tesis de maestr\'{\i}a y doctorado y especializaciones y especialidades en el \'{a}rea de la salud, desarrolladas en la Universidad Nacional de Colombia.\\
%
%Las m\'{a}rgenes, numeraci\'{o}n, tama\~{n}o de las fuentes y dem\'{a}s aspectos de formato, deben ser conservada de acuerdo con esta plantilla, la cual esta dise\~{n}ada para imprimir por lado y lado en hojas tama\~{n}o carta. Se sugiere que los encabezados cambien seg\'{u}n la secci\'{o}n del documento (para lo cual esta plantilla esta construida por secciones).\\
%
%Si se requiere ampliar la informaci\'{o}n sobre normas adicionales para la escritura se puede consultar la norma NTC 1486 en la Base de datos del ICONTEC (Normas T\'{e}cnicas Colombianas) disponible en el portal del SINAB de la Universidad Nacional de Colombia\footnote{ver: www.sinab.unal.edu.co}, en la secci\'{o}n "Recursos bibliogr\'{a}ficos" opci\'{o}n "Bases de datos".  Este portal tambi\'{e}n brinda la posibilidad de acceder a un instructivo para la utilizaci\'{o}n de Microsoft Word y Acrobat Professional, el cual est\'{a} disponible en la secci\'{o}n "Servicios", opci\'{o}n "Tr\'{a}mites" y enlace "Entrega de tesis".\\
%
%La redacci\'{o}n debe ser impersonal y gen\'{e}rica. La numeraci\'{o}n de las hojas sugiere que las p\'{a}ginas preliminares se realicen en n\'{u}meros romanos en may\'{u}scula y las dem\'{a}s en n\'{u}meros ar\'{a}bigos, en forma consecutiva a partir de la introducci\'{o}n que comenzar\'{a} con el n\'{u}mero 1. La cubierta y la portada no se numeran pero si se cuentan como p\'{a}ginas.\\
%
%Para trabajos muy extensos se recomienda publicar m\'{a}s de un volumen. Se debe tener en cuenta que algunas facultades tienen reglamentada la extensi\'{o}n m\'{a}xima de las tesis  o trabajo de investigaci\'{o}n; en caso que no sea as\'{\i}, se sugiere que el documento no supere 120 p\'{a}ginas.\\
%
%No se debe utilizar numeraci\'{o}n compuesta como 13A, 14B \'{o} 17 bis, entre otros, que indican superposici\'{o}n de texto en el documento. Para resaltar, puede usarse letra cursiva o negrilla. Los t\'{e}rminos de otras lenguas que aparezcan dentro del texto se escriben en cursiva.\\

%%% Importancia de la investigación %%%
Oil reservoir simulation has proven an usefull toll for predicting reserves and production along the years. The simulation of such a problem consists of solving a coupled set of mass balance equations across a domain (reservoir, geometry - geological model). Therefore, the creation of oil reservoir simulators is in the scietific software research area (Rewrite). 

Oil reservoir simulation consists of solving a set of coupled mass or moles balance equations, these equations are non-linear and need adecuate treatment in order to have a linear system that converges.

Since the natural production is no longer maintainable, techniques of enhanced oil recovery (EOR) have been developed in order to mantain or even improve the recovery factor. These techniques involve the injection of chemicals that affect the rock and fluids properties making feasible to change the oil mobility and residual saturations... The EOR processes add new equations to the system that make the problem even bigger. Many authors have addressed this problem by making general flow simulation frameworks. Those frameworks implement the general workflow of solving the coupled set of equations generated by the phenomena in the reservoir.

Some efforts have been done in the scientific software representation. Nore\~{a} et al. extend the preconceptual schema syntax defined by Zapata, 2007. for taking into account the elements needed in the scientific software context. Chaverra, 2011 includes cycles and conditional selection in the preconcpetual schema. Calle, 2017 defines design patterns in the context of scientific software using preconceptual schemas also extending its syntax.


The existing frameworks vary in implementation, even though they apply the same techniques. This is due to the fact that the design desitions are delegated to the programmer, which is an expert of flow in porous media simulation. Little effort has been done in representing the domain of reservoir simulation as is, including both dynamics and structure in the same representation. The existing studies in oil reservoir simulation domain representation lack of grouping the structural design with the dynamical behaviour. Others implement directly a solution of the set of differential equations for the specific study case. The problem knowledge is not shareable. The representations existing only account for the structural or, exclusively the dynamical behavior of the tool they developed. The use of the concepts lacks generality. Even though the formal definition of the differential equations, they lack information of constitutive equations. 

In this thesis we propose an event based representation of the enhanced oil recovery simulation using preconceptual schemas. In order to do that, For this purpose. we sizas sizas sizas describe the black oil simulation domain in the preconceptual schema syntax, later we define a generic component with variable kinetical behavior and with the capacity to change the flow properties in each phase.

The developed model couples the models used for an enhanced oil recovery process in a preconceptual schema that represents adecuately the oil reservoir simulation domain, the representation is validated with the SPE Comparative solution project having accordance with the reported results.